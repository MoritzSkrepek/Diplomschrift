\chapter{Kurzfassung}
Diese Diplomarbeit widmet sich der Entwicklung einer Lernapplikation für die HTL Wiener Neustadt unter Verwendung
der Unity-Plattform. Die Umsetzung erfolgte in Form einer Augmented Reality (AR) Applikation, speziell für die
Microsoft HoloLens 2.

Die Applikation besteht aus drei verschiedenen Szenarien. Darunter, dass Hauptmenu, das Nachrichtenaustausch und Knapsack-Problem
Anwendungsszenario,

Die Applikation ermöglicht es den Schülern, während des Tages der offenen Tür zwei wesentliche Grundprinzipien der
Informatik mithilfe von Augmented Reality auf spielerische und interessante Weise zu erkunden. Dies bietet den
Schülern die Möglichkeit zu erfahren, ob sie ein Interesse an solchen Themen haben. Auch wird darauf abgezielt diese
Applkation im Regelunterricht zu verwenden um so diese Prinzipien / Abläufe selbst anwenden und ausprobieren zu können.
