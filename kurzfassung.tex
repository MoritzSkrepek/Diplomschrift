\chapter{Kurzfassung}
Diese Abschlussarbeit widmet sich der Entwicklung einer Lernapplikation für die HTL Wiener Neustadt unter Verwendung
der Unity-Plattform. Die Umsetzung erfolgte in Form einer augmented reality (AR) Applikation, speziell für die
Microsoft HoloLens 2.

Die Applikation besteht aus drei verschiedenen Levels. Darunter, dass Hauptmenu, das Ping-Level und das Knapsack-Problem-level,
welche in Unity implementiert wurden.

Die Applikation ermöglicht es den Schülern, während des Tages der offenen Tür zwei wesentliche Grundprinzipien der
Informatik mithilfe von Augmented Reality auf spielerische und interessante Weise zu erkunden. Dies bietet den
Schülern die Möglichkeit zu erfahren, ob sie ein Interesse an solchen Themen haben. Der Einsatz von Unity als
Entwicklungsplattform ermöglichte eine umfassende und wissenschaftlich fundierte Umsetzung dieses Projekts.


