\chapter{Grundlagen}
In diesem Kapitel werden das Vorgehensmodell und alle Tools, die für die erfolgreiche
Abwicklung des Projekts nötig sind, erläutert.

\section{Vorgehensmodelle}
Im Vorfeld der Durchführung des Projekts wurden Informationen über diverse Vorgehensmodelle
gesammelt. Für das Projektteam war schnell klar, dass ein agiles Modell gewählt werden sollte,
da somit das Projekt dynamischer geplant und durchgeführt werden kann. Die Auswahl für Scrum
stand direkt bei Projektbegin fest. In dem folgenden Abscnhitt wird dieses
Vorgehendsmodell genauer erklärt und unsere Entscheidung anschließend begründet.

\section{Scrum}
Scrum\footnote{Quelle: Scrum Alliance \cite{WHAT-IS-SCRUM}} repräsentiert ein agiles Projektmanagement-Framework,
das auf die effiziente Entwicklung von Produkten und Software abzielt. Es legt besonderen Wert auf Zusammenarbeit,
Anpassungsfähigkeit und die kontinuierliche Bereitstellung funktionsfähiger Inkremente innerhalb kurzer Entwicklungszyklen,
den sogenannten Sprints. \\

Die zuvor skizzierte Definition gewährt einen knappen Einblick in das agile Vorgehensmodell Scrum. Die herausragenden Merkmale dieses Modells sind:

\begin{itemize}
    \item Drei zentrale Rollen, die im Folgenden näher erläutert werden.
    \item Der Product Backlog, der sämtliche Anforderungen enthält.
    \item Eine iterative und zeitlich definierte Entwicklung von Produkten.
    \item Die autonome Arbeitsweise des Teams.
    \item Gleichberechtigung aller Teammitglieder.
\end{itemize}

\subsection{Die drei Rollen in Scrum}
\begin{itemize}
    \item \textbf{Product Owner}\footnote{Scrum-Rolle \cite{Product-Owner}}: Der Product Owner trägt die Verantwortung
    für die Pflege des Product Backlogs und vertritt dabei die fachliche Auftraggeberseite sowie sämtliche Stakeholder.
    Ein zentrales Anliegen ist die Priorisierung der Elemente im Product Backlog, um den geschäftlichen Wert des
    Produkts zu maximieren und die Möglichkeit für frühe Veröffentlichungen essenzieller Funktionalitäten zu schaffen.
    Der Product Owner nimmt nach Möglichkeit an den täglichen Scrum-Meetings teil, um auf passive Weise Einblicke zu
    gewinnen. Zudem steht er dem Team für Rückfragen zur Verfügung, um einen reibungslosen Informationsaustausch zu gewährleisten.
    \item \textbf{Scrum Master}\footnote{Scrum-Rolle \cite{Scrum-Master}}: Der Scrum Master übernimmt eine zentrale
    Rolle im Scrum-Prozess und ist für die korrekte Umsetzung desselben verantwortlich. Als Vermittler und Unterstützer
    fungiert er als Facilitator, der darauf abzielt, einen maximalen Nutzen zu erzielen und kontinuierliche Optimierung
    sicherzustellen. Ein zentrales Anliegen ist die Beseitigung von Hindernissen, um ein reibungsloses Voranschreiten
    des Teams zu gewährleisten. Der Scrum Master sorgt für einen effizienten Informationsfluss zwischen dem Product Owner
    und dem Team, moderiert Scrum-Meetings und behält die Aktualität der Scrum-Artefakte wie Product Backlog,
    Sprint Backlog und Burndown Charts im Blick. Darüber hinaus liegt in seiner Verantwortung, das Team vor
    unberechtigten Eingriffen während des Sprints zu schützen.
    \item \textbf{Team}\footnote{Scrum-Rolle \cite{Team}}: Das Team, bestehend aus vier bis zehn Mitgliedern,
    idealerweise sieben, zeichnet sich durch eine interdisziplinäre Zusammensetzung aus, die Entwickler, Architekten,
    Tester und technische Redakteure einschließt. Durch Selbstorganisation agiert das Team eigenständig und übernimmt
    die Verantwortung als sein eigener Manager. Es besitzt die Befugnis, autonom über die Aufteilung von Anforderungen
    in Aufgaben zu entscheiden und diese auf die einzelnen Mitglieder zu verteilen, wodurch der Sprint Backlog aus dem
    aktuellen Teil des Product Backlog entsteht.
\end{itemize}

\noindent Alle Anforderungen an das Produkt werden in sogenannten User Stories, vorrangig erstellt durch den Product
Owner, im Product Backlog gesammelt. In einem Intervall, bezeichnet als Sprint, werden die User Stories abgearbeitet.
Die Projektentwicklung nach Scrum besteht aus fünf zentralen Elementen:
\begin{itemize}
    \item \textbf{Sprint: Planning Meeting}\footnote{Scrum-Meetings \cite{Sprint-planing-meeting}}: Im Sprint Planning
    Meeting wird das Ziel des folgenden Sprints definiert. Hierbei werden die Anforderungen im Project Backlog, die in
    diesem Sprint umgesetzt werden sollen, in einzelne Aufgaben zerlegt und anschließend im Sprint Backlog gesammelt.
    \item \textbf{Sprint}\footnote{Scrum-Meetings \cite{Sprint}}: Ein Sprint repräsentiert eine Entwicklungsphase,
    während der eine voll funktionsfähige und potenziell veröffentlichte Software entsteht. Die Dauer eines solchen
    Sprints beträgt typischerweise zwischen 1 und 4 Wochen.
    \item \textbf{Daily Scrum}\footnote{Scrum-Meetings \cite{Daily-Scrum}}: Der Daily Scrum ist ein kurzes Teammeeting,
    in dem Teammitglieder darüber informieren, welche Aufgaben seit dem letzten Meeting abgeschlossen wurden, woran bis
    zum nächsten Meeting gearbeitet werden muss und wo momentane Probleme existieren. Auf diese Weise sind alle Teammitglieder
    stets auf dem aktuellen Stand, was die Lösung aufkommender Probleme erleichtert.
    \item \textbf{Sprint Review}\footnote{Scrum-Meetings \cite{Sprint-Review}}: In diesem Meeting präsentiert das
    Entwicklungsteam die im Sprint abgeschlossenen Arbeitsergebnisse, beispielsweise fertige Produktinkremente, den
    Stakeholdern, zu denen Produktbesitzer, Kunden, Führungskräfte und andere Interessengruppen gehören.
    \item \textbf{Sprint Retrospective}\footnote{Scrum-Meetings \cite{Sprint-Retroperspektiv}}: Die Sprint Retrospective
    dient primär dazu, dass das Scrum-Team (bestehend aus dem Entwicklungsteam, dem Scrum Master und dem Product Owner)
    gemeinsam den abgeschlossenen Sprint reflektiert und Möglichkeiten zur kontinuierlichen Verbesserung identifiziert.
\end{itemize}

\noindent Durch diese Elemente kann ein optimaler Projektablauf gewährleistet werden. Das Projekt bleibt jederzeit
offen für Änderungen, und durch eine enge Zusammenarbeit mit dem Kunden können Missverständnisse und Probleme
frühzeitig behandelt und kommuniziert werden.

\subsection{Begründung der Auswahl}
Die Applied Augmented Reality in Education Applikation besteht aus 3 verschiedenen Level.
Im Team welches aus vier Schülern bestand übernahm jede Person einen Teilbereich oder arbeiteten
gemeinsam an einem dieser Level mit Unteraufgaben in diesem Level. Unterstützt wurde man von einem
Lehrer, der stetz für Fragen bereitstand und oftmals in beratender Form vorhanden war. Als
Vorgehensmodell wählte das Team das agile Modell Scrum. Die von Scrum gegebenen Richtlinien
konnten leicht eingehalten werden, da das Team täglich in der Schule aufeinander
traf als auch privat Kontakt hatten. Jederart Änderung, Problem oder Änderungen und anderartige
Dinge konnten daher leicht kommuniziert und besprochen werden. Am Ende jedes Sprints wurden
die erreichten Ergebnisse mit dem Betreuer besprochen, sowie die Neuerungen vorgestellt.
In den Sprintreviews konnte somit Feedback zu den Ergebnissen gesammelt werden und von dem
Betreuer konnten neue Ansichten und Denkweisen angebracht und integriert werden.
Durch die Sprint Retroperspektive konnten die Schüler einen größeren Mehrwert aus der
Projektentwicklung schöpfen, da sie neben der Verwendung des Scrum-Prozesses auch ihre Fähigkeiten
in den einzelnen Bereichen, durch das Besprechen der positiven und negativen Aspekte verbessern.

\section{Projektmanagement-Tools}
Um einen positiven Verlauf des Projekts zu ermöglichen, benötigt man die unterstützenden
Tools beim Projektmanagement sowie die Verwaltung von Dateien.

\subsection{GitHub}
Als sogenanntes Repository für die Source Code Dateien wurde GitHub mit der dazugehörigen
Webanwendung verwendet. Zu Beginn des Projekts stand die Entscheidung an, welche Technologie
und welcher Anbieter für das Versionskontrollsystem gewählt werden sollten. Neben GitHub gibt es
andere namhafte Anbieter solcher Verwaltungssysteme, darunter GitLab und SourceForge.

Ausschlaggebend für die Wahl von GitHub waren mehrere Punkte. Zum einen ist GitHub eine
kostenlose Lösung, die es ermöglicht, ein privates Projekt mit mehreren Mitgliedern
ohne Kosten anzulegen. Im Gegensatz dazu bieten manche Plattformen nur eine begrenzte Anzahl
von Mitgliedschaften in kostenfreien Projekten an. Die Registrierung erforderte lediglich
einen Account.

Darüber hinaus bietet GitHub eine benutzerfreundliche Oberfläche, eine breite Unterstützung
für verschiedene Programmiersprachen und eine aktive Entwicklergemeinschaft. Dies erleichtert
die Zusammenarbeit und den Informationsaustausch im Projektteam.

\subsection{Jira}
Als sogenanntes Verwaltungstool für die Vorgänge im Projekt wurde Jira mit der dazugehörigen
Webanwendung verwendet. Auch hier stand zu Projektbeginn die Frage im Raum, welche Technologie
und welcher Anbieter für das Aufgabenmanagement gewählt werden sollten. Neben Jira gibt es
weitere namhafte Anbieter solcher Tools, darunter VivifyScrum und KanBan.

Die Wahl von Jira basierte auf mehreren Überlegungen. Zum einen ist Jira eine kostenlose
Lösung, die es ermöglicht, ein SCRUM Board mit mehreren Mitgliedern kostenfrei anzulegen.
Ein weiterer entscheidender Faktor war die direkte Verbindung zu dem GitHub-Repository und die Möglichkeit,
neue Branches und Commits direkt in Jira zu erstellen.

Darüber hinaus bietet Jira eine umfassende Funktionalität für das Projektmanagement, einschließlich
der Verfolgung von Aufgaben, der Planung von Sprints und der Erstellung von Berichten. Diese Features
ermöglichen es dem Projektteam, den Fortschritt genau zu überwachen und eventuelle Herausforderungen
frühzeitig zu identifizieren und anzugehen.

\section{Konzeption von Fragebögen}
Bei jeder Umfrage werden Informationen von Personen oder Personengruppen zu der allgemeinen
Umsetzung und dem Verständis der Applikation gesammelt. Diese werden im Anschluss ausgewertet und
interpretiert. Wichtig ist hier den Zweck jeder Umfrage genau zu definieren. Durch präzise und
detailierte Zielsetzungen ist es später dann möglich, den Erfolg der Umfrage zu garantieren.

\subsection{Planung der Fragebogenkonzeption}
Die Konzeption und Gestalltung eines Fragebogens ist der wichtigste Schritt bei der Planung.
Eine gut überlegte Planungphase führt zu besseren Ergebnissen und dadurch auch eine leichtere
Evaluierung. Folgende Entscheidung müssen daher schon im Vorfeld definiert und getroffen werden:
\begin{itemize}
    \item Inhalt: evtl. bestehende Fragebögen verwenden oder anpassen.
    \item Umfang: Eher kurz halten (In Abhängigkeit von den Zielen).
    \item Ablauf und zeitlicher Rahmen: postalisch (längere Rücklaufzeit) oder elektronisch \footnote{Vgl. \cite{Buehner} S. -}
    \item Zielgruppe: Vollbefragung oder Stichproben \footnote{Vgl. \cite{Mayer} S. -}
\end{itemize}

\subsection{Abfassung der Fragen}
Der Erfolg einer Umfrage benötigt eine genau Vorbereitung. Im Vorfeld muss klar sein,
dass nur einzelne Auschnitte eines Themengebietes behandelt werden können. Diese Ausschnitte
müssen daher umso enger und genauer definiert werden. Hier ist daher vorallem die eindeutige
Formulierung der Fragen wichtig.

im Vordergrund bei der Fragenformulierung stehen hier die Verständlichkeit bzw. die Unmissverständlichkeit.
Folgende Regeln zur Formulierung sollen daher eingehalten werden:
\begin{itemize}
    \item Einfache Wörter: Wörte, keine Fachausdrücke, andersprachige Wörter oder Fremdwörter
    \item Formulierung: Möglichst kurz
    \item Keine belastenden Wörter verwenden (z.B.: Ehrlichkeit, etc...)
    \item Keine hypothetischen Formulierungen
    \item Nur auf einen bestimmten Sachverhalt beziehen
    \item Keine Überforderung (Nicht zu viele Informationen auf einmal)
    \item Keine doppelten Verneinungen \footnote{Vgl. \cite{Mayer} S. -}
\end{itemize}

Diese Kriterien gelten für eine schriftliche Befragung. Um das Resultat dieser Umfrage nicht
zu verfälschen darf der Interviewer keine Extrafragen oder Umformulierungen an den gestellten
Fragen tätigen.

\subsection{Arten von Fragen}
Je nach Anforderungsbedinungen wird zwischen einer offenen und geschlossenen Frage unterschieden \footnote{Vgl. \cite{Mayer} S. -}

%Offene und geschlossene Fragen erklären

\subsection{Struktur und Gliederung von Fragebögen}
Hier wird verfasst wie die Allgemeine Struktur und Gliederung von Fragebögen aussehen soll. \footnote{Vgl. \cite{Buehner} S. -}

\subsection{Mögliche Verfälschung des Resultats}
Welche Arten von Verfälschungen gibt es und diese Beschreiben
Ursachen dafür beschreiben. \footnote{Vgl. \cite{Buehner} S. -}

\subsection{Auswertung von Fragebögen}
Wie werten wir die Fragenbögen aus? \footnote{Vgl. \cite{Mayer} S. -}

%Quellen zu den Fragenbögen Dingen finden und ALLES genauer beschreiben
