\chapter{Grundlagen}
In diesem Kapitel werden das Vorgehensmodell und alle Tools, die für die erfolgreiche
Abwicklung des Projekts nötig sind, erläutert.

\section{Vorgehensmodelle}
Im Vorfeld der Durchführung des Projekts wurden Informationen über diverse Vorgehensmodelle
gesammelt. Für das Projektteam war schnell klar, dass ein agiles Modell gewählt werden sollte,
da somit das Projekt dynamischer geplant und durchgeführt werden kann. Die Auswahl für Scrum
stand direkt bei Projektbegin fest. In dem folgenden Abscnhitt wird dieses
Vorgehendsmodell genauer erklärt und unsere Entscheidung anschließend begründet.

\subsection{Scrum}
Scrum \footnote{Scrum Alliance \cite{WHAT-IS-SCRUM}} ist ein agiles Projektmanagement-Framework zur effizienten Entwicklung von Produkten und Software.
Es betont Zusammenarbeit, Anpassungsfähigkeit und Lieferung von Arbeitsfähigem in kurzen Zeitspannen (Sprints). \\

%Source: https://www.scrumalliance.org/about-scrum#!section1
\noindent Die oben angeführte Definition gibt einen kurzen Einblick in das agile Vorgehendsmodell
Scrum. Zu den Hauptmerkmalen dieses Modells zählen folgende:
\begin{itemize}
    \item Drei Rollen, welche nachfolgend erklärt werden
    \item Product Backlog, welches Anforderungen enthält
    \item Produktentwicklung erfolgt iterativ und in zeitlich definierten Zyklen
    \item Das Team arbeitet autonom
    \item Alle Mitglieder sind gleichberechtigt\\
\end{itemize}

\subsubsection{Die drei Rollen in Scrum}
\begin{itemize}
    \item Product Owner \footnote{Scrum-Rolle \cite{Product-Owner}}: \\
    Die Pflege des Product Backlogs liegt in der Verantwortung dieser Rolle, die die fachliche Auftraggeberseite
    sowie sämtliche Stakeholders vertritt. Ein zentrales Anliegen ist die Priorisierung der Product Backlog
    Items in einer Weise, die den Business Value des Produkts maximal steigert und die Möglichkeit für
    frühzeitige Releases von Kernfunktionalitäten schafft. Dies ermöglicht einen schnellen Return on Investment.
    Um stets gut informiert zu sein, nimmt die Person nach Möglichkeit an den Daily Scrums teil, um auf passive
    Weise Einblicke zu gewinnen. Zudem steht sie dem Team für Rückfragen zur Verfügung, um einen reibungslosen
    Informationsaustausch zu gewährleisten.
    \item Scrum Master \footnote{Scrum-Rolle \cite{Scrum-Master}}: \\
    Der Scrum-Master übernimmt eine zentrale Rolle im Scrum-Prozess und ist für die korrekte Umsetzung desselben verantwortlich.
    Als Vermittler und Unterstützer fungiert er als Facilitator, der darauf abzielt, einen maximalen Nutzen zu erzielen
    und kontinuierliche Optimierung sicherzustellen. Ein zentrales Anliegen ist die Beseitigung von Hindernissen, um
    ein reibungsloses Voranschreiten des Teams zu gewährleisten. Der ScrumMaster sorgt für einen effizienten
    Informationsfluss zwischen dem Product Owner und dem Team, moderiert Scrum-Meetings und behält die Aktualität
    der Scrum-Artefakte wie Product Backlog, Sprint Backlog und Burndown Charts im Blick. Darüber hinaus liegt in
    seiner Verantwortung, das Team vor unberechtigten Eingriffen während des Sprints zu schützen.
    -Werte zu verstehen.
    \item Team \footnote{Scrum-Rolle \cite{Team}}: \\
    Das Team, bestehend aus fünf bis zehn Personen (idealerweise sieben), zeichnet sich durch seine interdisziplinäre
    Zusammensetzung aus, die Entwickler, Architekten, Tester und technische Redakteure umfasst – eine Struktur, die in
    den meisten Fällen von Vorteil ist. Durch Selbstorganisation agiert das Team eigenständig und übernimmt die
    Verantwortung als sein eigener Manager. Es hat die Befugnis, autonom über die Aufteilung von Anforderungen in
    Aufgaben zu entscheiden und diese auf die einzelnen Mitglieder zu verteilen, wodurch der Sprint Backlog aus dem
    aktuellen Teil des Product Backlog entsteht.\\
\end{itemize}
\\
\noindent Alle Anforderungen an das Produkt werden in sogennanten User Stories, die hauptsächlich
vom Product Owner erstellt werden, im Product Backlog gesammelt. in einem Intervall, welcher
als Sprint bezeichnet wird, werden die User Stories abgearbeitet. Die Projektentwicklung
nach Scrum besteht dabei aus 5 Elementen:
\begin{itemize}
    \item Sprint: Planning Meeting \footnote{Scrum-Meetings \cite{Sprint-planing-meeting}}\\
    Im Sprint Planning Meeting wird das Ziel des folgendes Sprints festgelegt. Hier werden
    jene Anforderungen im Projekt Backlog, die in diesem Sprint realisiert werden sollen,
    in einzelne Aufgaben zerteilt und anschließend im Sprint Backlog gesammelt.
    \item Sprint: \footnote{Scrum-Meetings \cite{Sprint}} \\
    Ein Sprint ist eine Entwicklungsphase, in welcher eine vollfunktionsfähige und potentielle
    veröffentlichbare Software entsteht. Die Dauer eines solchen Sprints liegt zwischen
    1 bis 4 Wochen.
    \item Daily Scrum: \footnote{Scrum-Meetings \cite{Daily-Scrum}}\\
    Der Daily Scrum ist ein kurzes Teammeeting in dem Teammitglieder mitteilen, welche
    Aufgaben seit dem letzten Daily Scrum abgeschlossen wurden, woran bis zum nächsten
    Daily Scrum gearbeitet werden muss und wo momentan Probleme existieren. Somit sind alle
    Teammitglieder stätig up-to-date und kennen den aktuellen Stand wodurch anstehende Probleme
    leichter gelöst werden können.
    \item Sprint Review: \footnote{Scrum-Meetings \cite{Sprint-Review}}\\
    In diesem Meeting präsentiert das Entwicklungsteam die in diesem Sprint abgeschlossenen
    Arbeitsergebnisse (z. B. fertige Produktinkremente) den Stakeholdern, dies können
    Produktbesitzer, Kunden, Führungskräfte und andere Interessengruppen sein.
    \item Sprint Retroperspektive: \footnote{Scrum-Meetings \cite{Sprint-Retroperspektiv}} \\
    Ihr Hauptzweck besteht darin, dass das Scrum-Team (das Entwicklungsteam, der Scrum-Master
    und der Produktbesitzer) gemeinsam über den abgeschlossenen Sprint reflektiert und
    Möglichkeiten zur kontinuierlichen Verbesserung identifiziert.
\end{itemize}
\noindent Durch diese Elemente kann ein optimaler Projektablauf gewehrleistet werden. Das
Projekt ist jederzeit für Änderungen offen und durch eine enge Zusammenarbeit mit dem Kunden
können Missverständnisse und Probleme früh behandelt und kommuniziert werden.

\subsection{Begründung der Auswahl}
Die Applied Augmented Reality in Education Applikation besteht aus 3 verschiedenen Level.
Im Team welches aus vier Schülern bestand übernahm jede Person einen Teilbereich oder arbeiteten
gemeinsam an einem dieser Level mit Unteraufgaben in diesem Level. Unterstützt wurde man von einem
Lehrer, der stetz für Fragen bereitstand und oftmals in beratender Form vorhanden war. Als
Vorgehensmodell wählte das Team das agile Modell Scrum. Die von Scrum gegebenen Richtlinien
konnten leicht eingehalten werden, da das Team täglich in der Schule aufeinander
traf als auch privat Kontakt hatten. Jederart Änderung, Problem oder Änderungen und anderartige
Dinge konnten daher leicht kommuniziert und besprochen werden. Am Ende jedes Sprints wurden
die erreichten Ergebnisse mit dem Betreuer besprochen, sowie die Neuerungen vorgestellt.
In den Sprintreviews konnte somit Feedback zu den Ergebnissen gesammelt werden und von dem
Betreuer konnten neue Ansichten und Denkweisen angebracht und integriert werden.
Durch die Sprint Retroperspektive konnten die Schüler einen größeren Mehrwert aus der
Projektentwicklung schöpfen, da sie neben der Verwendung des Scrum-Prozesses auch ihre Fähigkeiten
in den einzelnen Bereichen, durch das Besprechen der positiven und negativen Aspekte verbessern.

\section{Projektmanagement Tools}
Um einen positiven Verlauf des Projekts zu ermöglichen, benötigt man die unterstützenden
Tolls beim Projektmanagement sowie die Verwaltung von Dateien.

\subsection{GitHub}
Als sogennantes Repository für die Source Code Dateien wurde GitHub mit der dazugehörigen
Webanwendung verwendet. Hier stand am Anfang des Projekt die Frage welche Technologie und
welcher Anbieter gewählt werden soll. Andere namhafte Anbieter solche Verwaltungssystem sind:
\begin{itemize}
    \item GitLab
    \item SourceForge
\end{itemize}
Ausschlaggebend für die Wahl von GitHub waren mehrere Punkte. Einerseits ist GitHub eine
kostenlose Lösung. Das bedeutet, dass man gratis ein privates Projekt mit mehreren Mitgliedern
anlegen kann. Manche Lösungen bieten hier beispielsweise nur eine begrenzte Anzahl von
Mitgliedern an. Benötigt wurde lediglich ein Account zur Registration.

\subsection{Jira}
Als sogennantes Verwaltungstool für die Vorgänge in dem Project wurde Jira mit der dazugehörigen
Webanwendung verwendet. Hier stand am Anfang des Projekts ebenfalls die Frage welche Technologie
und welcher Anbieter gewählt werden soll. Andere namhafte Anbieter solcher Tools sind:
\begin{itemize}
    \item VivifyScrum
    \item KanBan
\end{itemize}
Ausschlaggebend für die Wahl von Jira waren mehrere Punkte. Einersetis ist Jira eine kostenlose
Lösung. Das bedeutet, dass man ein SCRUM Board mit mehreren Mitgleidern gratis anlegen kann.
Ein weiterer Punkt ist die direkte Verbindung zu dem GitHub Repository und die Möglichkeit,
dass in Jira selbst neue Branches und Commits auf das Repository erstellt werden können.

\section{Konzeption von Fragebögen}
Bei jeder Umfrage werden Informationen von Personen oder Personengruppen zu der allgemeinen
Umsetzung und dem Verständis der Applikation gesammelt. Diese werden im Anschluss ausgewertet und
interpretiert. Wichtig ist hier den Zweck jeder Umfrage genau zu definieren. Durch präzise und
detailierte Zielsetzungen ist es später dann möglich, den Erfolg der Umfrage zu garantieren.

\subsection{Planung der Fragebogenkonzeption}
Die Konzeption und Gestalltung eines Fragebogens ist der wichtigste Schritt bei der Planung.
Eine gut überlegte Planungphase führt zu besseren Ergebnissen und dadurch auch eine leichtere
Evaluierung. Folgende Entscheidung müssen daher schon im Vorfeld definiert und getroffen werden:
\begin{itemize}
    \item Inhalt: evtl. bestehende Fragebögen verwenden oder anpassen.
    \item Umfang: Eher kurz halten (In Abhängigkeit von den Zielen).
    \item Ablauf und zeitlicher Rahmen: postalisch (längere Rücklaufzeit) oder elektronisch \footnote{Vgl. \cite{Buehner} S. -}
    \item Zielgruppe: Vollbefragung oder Stichproben \footnote{Vgl. \cite{Mayer} S. -}
\end{itemize}

\subsection{Abfassung der Fragen}
Der Erfolg einer Umfrage benötigt eine genau Vorbereitung. Im Vorfeld muss klar sein,
dass nur einzelne Auschnitte eines Themengebietes behandelt werden können. Diese Ausschnitte
müssen daher umso enger und genauer definiert werden. Hier ist daher vorallem die eindeutige
Formulierung der Fragen wichtig.

im Vordergrund bei der Fragenformulierung stehen hier die Verständlichkeit bzw. die Unmissverständlichkeit.
Folgende Regeln zur Formulierung sollen daher eingehalten werden:
\begin{itemize}
    \item Einfache Wörter: Wörte, keine Fachausdrücke, andersprachige Wörter oder Fremdwörter
    \item Formulierung: Möglichst kurz
    \item Keine belastenden Wörter verwenden (z.B.: Ehrlichkeit, etc...)
    \item Keine hypothetischen Formulierungen
    \item Nur auf einen bestimmten Sachverhalt beziehen
    \item Keine Überforderung (Nicht zu viele Informationen auf einmal)
    \item Keine doppelten Verneinungen \footnote{Vgl. \cite{Mayer} S. -}
\end{itemize}

Diese Kriterien gelten für eine schriftliche Befragung. Um das Resultat dieser Umfrage nicht
zu verfälschen darf der Interviewer keine Extrafragen oder Umformulierungen an den gestellten
Fragen tätigen.

\subsection{Arten von Fragen}
Je nach Anforderungsbedinungen wird zwischen einer offenen und geschlossenen Frage unterschieden \footnote{Vgl. \cite{Mayer} S. -}

%Offene und geschlossene Fragen erklären

\subsection{Struktur und Gliederung von Fragebögen}
Hier wird verfasst wie die Allgemeine Struktur und Gliederung von Fragebögen aussehen soll. \footnote{Vgl. \cite{Buehner} S. -}

\subsection{Mögliche Verfälschung des Resultats}
Welche Arten von Verfälschungen gibt es und diese Beschreiben
Ursachen dafür beschreiben. \footnote{Vgl. \cite{Buehner} S. -}

\subsection{Auswertung von Fragebögen}
Wie werten wir die Fragenbögen aus? \footnote{Vgl. \cite{Mayer} S. -}

%Quellen zu den Fragenbögen Dingen finden und ALLES genauer beschreiben
