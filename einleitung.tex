\chapter{Einleitung}
\label{cha:Einleitung}

\section{Ausgangslage}
Um dem IT-Fachkräftemangel entgegenzuwirken, ist es entscheidend, die Ausbildung im MINT-Bereich attraktiver zu gestalten.
Diese Diplomarbeit zielt darauf ab, einen wichtigen Beitrag zu diesem Ziel zu leisten, unterstützt durch das Förderprogramm
\textit{Wissenschaft trifft Schule} des Landes Niederösterreich. Dabei sollen exemplarische Anwendungen im Bereich Augmented
Reality für die Vermittlung von Informatik-Lehrinhalten evaluiert und umgesetzt werden.

\section{Auslöser}
Die Besucher des \textit{Tag der offenen Tür} erhalten durch diese Applikation eine Vorführung der neuesten Technologien,
was dazu beiträgt, dass sie erkennen, dass die Schule einen sehr hohen Technologiestandard aufweist. Dies führt zu einer
signifikanten Steigerung der Nachfrage bei zukünftigen Bewerbungen für die Abteilung Informatiktechnik. Darüber hinaus
stärkt dies den Ruf der Schule nach außen und präsentiert sie als attraktiven Ausbildungsstandort für zukünftige Mitarbeiter
vieler Unternehmen.

\section{Aufgabenstellung}
Die Aufgabenstellung besteht darin, zwei Unity-Szenarien zu erstellen, die jeweils ein Grundprinzip der Informationstechnologie
veranschaulichen. Im ersten Szenario wird das Grundprinzip des \textit{Pingen} zwischen zwei PCs dargestellt. Mithilfe
der HoloLens 2 wird ein 3D-modelliertes Paket auf einem Kabel visualisiert, das zwei reale PCs miteinander verbindet. Um
das Senden und Empfangen eines Ping-Pakets zu simulieren, wird eine einfache Website mit einem entsprechenden Backend und
der Verwendung einer RESTful API erstellt, um Nachrichten zwischen den PCs zu senden und zu empfangen. Dies verdeutlicht
dem Benutzer den eigentlich unsichtbaren \textit{Ping} zwischen zwei verbundenen Geräten.

Im zweiten Szenario wird das in der IT bekannte und vielseitig verwendete Optimierungsproblem des \textit{Knapsack Problems}
dargestellt. Mithilfe von Augmented Reality wird ein Inventar auf einer Oberfläche platziert, die die verfügbare Fläche für
die Platzierung der Gegenstände repräsentiert. Die Gegenstände im Inventar sind typische, im Alltag verwendete Gegenstände
wie Laptop, Maus, Tastatur usw. Jeder dieser Gegenstände kann mithilfe eines QR-Codes visualisiert werden, um ein 3D-Modell
des Gegenstands und die zugehörigen Werte wie Gewicht und Wert für die Lösung des Knapsack-Problems anzuzeigen. Die Aufgabe
des Benutzers besteht darin, das Inventar bestmöglich mit den verfügbaren Gegenständen zu füllen, um den Gesamtwert des
Inventars zu maximieren. Zusätzlich zur Lösung des Problems kann der Benutzer durch einen Knopfdruck eine der berechneten
optimalen Lösungen aktivieren und sichtbar machen.

\section{Team}
Das Diplomarbeitsteam besteht aus:
\begin{itemize}
    \item Moritz SKREPEK
    \item Seref HAYLAZ
    \item Dustin LAMPEL
    \item Jonas SCHODITSCH
\end{itemize}

\subsection{Aufteilung}
Die Rolle des Projektsleiters der Diplomarbeit nahm Moritz SKREPEK ein, da dieser die Grundlegende Idee für die Darstellung
zweier IT-Grundprinzipien mittels der Microsoft HoloLens2 hatte.

Die entwickelte Applikation lässt sich in das Hauptmenü, Pinkt-Szenario und Knapsack-Problem-Szenario einglieder. Die Implementierung
des Hauptmenüs und ebenfalls für das gesamte UI/UX übernahm Dustin LAMPEL unter Verwendung der UX-Tools-Plugins für Mixed Reality.
Die Umsetzung des Pink-Paket-Szenarios übernahm SCHODITSCH Jonas mittels Objekt-Tracking und Picture-Taking. Für die Implementierung des
Knapsack-Problem-Szenarios waren Moritz SKREPEK und Seref HAYLAZ mittels Verwendung von Plane-detection,
QR-Code Tagging und Tracking, Knapsack Algorithmus, 3D Unity Game Objekte und Unittests zuständig.



