\chapter{Einleitung}
\label{cha:Einleitung}

\section{Ausgangslage}\marginpar{\small\(\rightarrow\) {\tiny SKREPEK}}
Um dem IT-Fachkräftemangel entgegenzuwirken, ist es entscheidend, die Ausbildung im MINT-Bereich attraktiver zu gestalten.
Diese Diplomarbeit zielt darauf ab, einen wichtigen Beitrag zu diesem Ziel zu leisten, unterstützt durch das Förderprogramm
\textit{Wissenschaft trifft Schule} des Landes Niederösterreich. Dabei sollen exemplarische Anwendungen im Bereich Augmented
Reality für die Vermittlung von Informatik-Lehrinhalten evaluiert und umgesetzt werden.

\section{Auslöser}
Die Besucher des \textit{Tag der offenen Tür} erhalten durch diese Applikation eine Vorführung der neuesten Technologien,
was dazu beiträgt, dass sie erkennen, dass die Schule einen sehr hohen Technologiestandard aufweist. Dies soll sicherstellen,
dass sich auch in Zukunft viele interessierte Schüler für eine Ausbildung an der Abteilung für Informatik entscheiden. Darüber
hinaus stärkt dies den Ruf der Schule nach außen und präsentiert sie als attraktiven Ausbildungsstandort für zukünftige Mitarbeiter
vieler Unternehmen.

\section{Aufgabenstellung}
Die Aufgabenstellung besteht darin, zwei Anwendungszenarien zu erstellen, die jeweils ein Grundprinzip der Informatik
veranschaulichen. Im ersten Szenario wird das Grundprinzip des Nachrichtenaustauschs zwischen zwei PCs dargestellt. Das Ziel
hierbei ist es dem Benutzer das eigentlich unsichtbare Nachrichtenpaket zu visualisieren. Der Benutzer soll hier auf einer
Website eine Nachricht eingeben können, um diese dann an den anderen Computer zu senden und zusätzlich kann auch der Inhalt
dieses Pakets durch draufdrücken visualisiert werden.

Im zweiten Szenario wird das in der IT bekannte und vielseitige verwendete Optimierungsproblem des Knapsack-Problems dargestellt.
Mit Hilfe von Augmented Reality wird der Rucksack als Inventar angezeigt in welchem der Benutzer alltägliche Gegenstände
platzieren kann. Die Aufgabe des Benutzers besteht darin, den Rucksack bestmöglich mit den verfügbaren Gegenständen
zu füllen, um den Gesamtwert den Rucksacks zu maximieren. Zusätzlich zur Lösung des Problems kann der Benutzer durch
einen Knopfdruck eine der berechneten perfekten Lösungen anzeigen.

\section{Team}
Das Diplomarbeitsteam besteht aus:
\begin{itemize}
    \item Moritz SKREPEK
    \item Seref HAYLAZ
    \item Dustin LAMPEL
    \item Jonas SCHODITSCH
\end{itemize}

\subsection{Aufteilung}
Die Rolle des Projektsleiters der Diplomarbeit nahm Moritz SKREPEK ein, da dieser die grundlegende Idee für die Darstellung
zweier Anwendungsszenarios mittels der Microsoft HoloLens2 hatte.

Die entwickelte Applikation lässt sich in das Hauptmenü, das Nachrichtenaustausch-Szenario und das Knapsack-Problem-Szenario
gliedern. Die Implementierung sowie die gesamte UI/UX und das Frontend der Website übernahm Dustin LAMPEL unter Verwendung
des UX-Tools-Plugins, welches sämtliche UI/UX Elemente für Mixed Reality Anwendungen bereitstellt und HTML, CSS und JavaScript.
Die Umsetzung des Nachrichtenaustausch-Szenarios übernahm SCHODITSCH Jonas mittels Objekt-Tracking und Foto-Aufnahmen. Für
die Implementierung des Knapsack-Problem-Szenarios waren Moritz SKREPEK und Seref HAYLAZ mittels Verwendung von Plane-detection,
QR-Code Tagging und Tracking, Knapsack Algorithmus und Unittests zuständig.



