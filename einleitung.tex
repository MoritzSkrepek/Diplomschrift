\chapter{Einleitung}
\label{cha:Einleitung}

\section{Ausgangslage}
Um dem IT-Fachkräftemangel entgegenzuwirken, muss die Ausbildung im
MINT-Bereich attraktiviert werden. Diese Diplomarbeit will hier, unterstützt
durch das Förderprogramm "Wissenschaft trifft Schule" des Landes NÖ, einen
wichtigen Beitrag leisten. Dazu sollen exemplarische Anwendungen im Bereich
Augmented Reality für die Vermittlung von Informatik-Lehrinhalten evaluiert und
umgesetzt werden.

\section{Auslöser}
Die Besucher des ”Tag der offenen Tür” bekommen mit dieser Applikation die neusten
Technologien vorgef¨uhrt und erkennen dadurch, dass die Schule sich auf einen sehr hohen
Technologiestandard befindet. Dadurch kommt es zu einer deutlich erhöhten Nachfrage
bei zukünftigen Bewerbungen für die Abteilung Informatitionstechnik. Weiters wird nach
Außen hin der Ruf der Schule gestärkt und diese präsentiert sich damit als attraktiver
Ausbildungsstandort für die zuk¨unftigen Mitarbeiter vieler Unternehmen.

\section{Aufgabenstellung}
Erstellen des Levelinhalts mit der Verwendung von 2 realen Laptops. Mit
Hilfe der HoloLens wird ein 3D modelliertes Ping Paket auf dem Kabel, dass
die zwei Laptops verbindet dargestellt. Wenn der Benutzer auf der Tastatur auf
die ”ENTER” Taste drückt, wird ein Ping Befehl ausgef¨uhrt und die modellierten
Pakete werden durch die HoloLens auf dem Netzwerkkabel dargestellt. Dies veranschaulicht
dem Benutzer den eigentlich nicht sichtbaren Ping von einem auf den anderen Laptop

\section{Team}
Das Diplomarbeitsteam besteht aus:
\begin{itemize}
    \item Moritz SKREPEK
    \item Seref HAYLAZ
    \item Dustin LAMPEL
    \item Jonas SCHODITSCH
\end{itemize}


\subsection{Aufteilung}
Die Rolle des Projektleiters der Diplomarbeit nahm Moritz SKREPEK ein, da dieser die
Grundidee für die Darstellung zweier IT-Grundprinzipien mittels der Microsoft HoloLens2
hatte.
Das Entwickelte System lässt sich in das Hauptmenu, das Ping-Paket-Level und das Knappsack-
Problem-Level gliedern. Die Implementierung des Hauptmenus übernahm Dustin LAMPEL, dabei
verwendete er für die UI/UX das UX-Tools-Plug-Ins für Mixed Reality. Die Umsetzung des
Ping-Paket-Levels übernahmen Seref HAYLAZ, Dustin LAMPEL und Jonas SCHODITSCH mittels
Object-Tracking, Kurvenberechnungen und 3D-Objekten. Das Knappsack-Problem-Level übernahmen
Moritz SKREPEK und Seref HAYLAZ mittels Verwendung von Spatial Anchors, Spatial Mapping, Qr-Code
Tracking, Knappsack-Algorithmus, 3D-Objekte und Unittests.
