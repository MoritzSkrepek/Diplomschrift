\usepackage{marginnote}
\chapter{Organisatorische Grundlagen}

\section{Vorgehensmodelle}
Im Vorfeld der Durchführung des Projekts wurden Informationen über diverse Vorgehensmodelle
gesammelt. Für das Projektteam war schnell klar, dass ein agiles Modell gewählt werden sollte,
da somit das Projekt dynamischer geplant und durchgeführt werden kann. Die Auswahl stand direkt
fest und wir entschieden uns für SCRUM.

\subsection{SCRUM}
Scrum, ein agiles Framework, revolutioniert Projektmanagement weltweit.
Kurze, 2- bis 4-wöchige Sprints bilden den Kern. Teams organisieren sich selbst,
unterstützt vom Scrum Master und geleitet vom Product Owner. Tägliche Stand-up-Meetings,
Reviews und Retrospektiven fördern Transparenz und Anpassungsfähigkeit. Scrum ermöglicht
inkrementelle Produktentwicklung und kontinuierliche Verbesserung. Dieses flexible
Konzept findet in vielen Branchen Anwendung und trägt zur Bewältigung der steigenden
Anforderungen an schnelle, adaptive Produktentwicklung bei.

\subsection{Warum SCRUM?}
SCRUM wird deshalb verwendet um eine gute Übersicht über das ganze Projekt zu behalten.
Außerdem um in gleichen Abständen Team-Meetings zu halten um mit den Team-Kameraden zu
kommunizieren wie der aktuelle Stand der Dinge ist.

\section{Projektmanagement Tools}
Um einen positiven Verlauf des Projekts zu ermöglichen, benötigt man die unterstützenden
Tolls beim Projektmanagement sowie die Verwaltung von Dateien.

\subsection{GitHub}
Als sogennantes Repository für die Source Code Dateien wurde GitHub mit der dazugehörigen
Webanwendung verwendet. Hier stand am Anfang des Projekt die Frage welche Technologie und
welcher Anbieter gewählt werden soll. Andere namhafte Anbieter solche Verwaltungssystem sind:
\begin{itemize}
    \item GitLab
    \item SourceForge
\end{itemize}
Ausschlaggebend für die Wahl von GitHub waren mehrere Punkte. Einerseits ist GitHub eine
kostenlose Lösung. Das bedeutet, dass man gratis ein privates Projekt mit mehreren Mitgliedern
anlegen kann. Manche Lösungen bieten hier beispielsweise nur eine begrenzte Anzahl von
Mitgliedern an. Benötigt wurde lediglich ein Account zur Registration.

\subsection{Jira}
Als sogennantes Verwaltungstool für die Vorgänge in dem Project wurde Jira mit der dazugehörigen
Webanwendung verwendet. Hier stand am Anfang des Projekts ebenfalls die Frage welche Technologie
und welcher Anbieter gewählt werden soll. Andere namhafte Anbieter solcher Tools sind:
\begin{itemize}
    \item VivifyScrum
    \item KanBan
\end{itemize}
Ausschlaggebend für die Wahl von Jira waren mehrere Punkte. Einersetis ist Jira eine kostenlose
Lösung. Das bedeutet, dass man ein SCRUM Board mit mehreren Mitgleidern gratis anlegen kann.
Ein weiterer Punkt ist die direkte Verbindung zu dem GitHub Repository und die Möglichkeit,
dass in Jira selbst neue Branches und Commits auf das Repository erstellt werden können.

\section{Konzeption von Fragebögen -> SKREPEK}
Bei jeder Umfrage werden Informationen von Personen oder Personengruppen zu der allgemeinen
Umsetzung und dem Verständis der Applikation gesammelt. Diese werden im Anschluss ausgewertet und
interpretiert. Wichtig ist hier den Zweck jeder Umfrage genau zu definieren. Durch präzise und
detailierte Zielsetzungen ist es später dann möglich, den Erfolg der Umfrage zu garantieren.

\subsection{Planung der Fragebogenkonzeption}
Die Konzeption und Gestalltung eines Fragebogens ist der wichtigste Schritt bei der Planung.
Eine gut überlegte Planungphase führt zu besseren Ergebnissen und dadurch auch eine leichtere
Evaluierung. Folgende Entscheidung müssen daher schon im Vorfeld definiert und getroffen werden:
\begin{itemize}
    \item Inhalt: evtl. bestehende Fragebögen verwenden oder anpassen.
    \item Umfang: Eher kurz halten (In Abhängigkeit von den Zielen).
    \item Ablauf und zeitlicher Rahmen: postalisch (längere Rücklaufzeit) oder elektronisch
    \item Zielgruppe: Vollbefragung oder Stichproben
\end{itemize}

\subsection{Abfassung der Fragen}
Der Erfolg einer Umfrage benötigt eine genau Vorbereitung. Im Vorfeld muss klar sein,
dass nur einzelne Auschnitte eines Themengebietes behandelt werden können. Diese Ausschnitte
müssen daher umso enger und genauer definiert werden. Hier ist daher vorallem die eindeutige
Formulierung der Fragen wichtig.

im Vordergrund bei der Fragenformulierung stehen hier die Verständlichkeit bzw. die Unmissverständlichkeit.
Folgende Regeln zur Formulierung sollen daher eingehalten werden:
\begin{itemize}
    \item Einfache Wörter: Wörte, keine Fachausdrücke, andersprachige Wörter oder Fremdwörter
    \item Formulierung: Möglichst kurz
    \item Keine belastenden Wörter verwenden (z.B.: Ehrlichkeit, etc...)
    \item Keine hypothetischen Formulierungen
    \item Nur auf einen bestimmten Sachverhalt beziehen
    \item Keine Überforderung (Nicht zu viele Informationen auf einmal)
    \item Keine doppelten Verneinungen
\end{itemize}

Diese Kriterien gelten für eine schriftliche Befragung. Um das Resultat dieser Umfrage nicht
zu verfälschen darf der Interviewer keine Extrafragen oder Umformulierungen an den gestellten
Fragen tätigen.


\subsection{Struktur und Gliederung von Fragebögen}
Hier wird verfasst wie die Allgemeine Struktur und Gliederung von Fragebögen aussehen soll.

\subsection{Mögliche Verfälschung des Resultats}
Welche Arten von Verfälschungen gibt es und diese Beschreiben
Ursachen dafür beschreiben.

\subsection{Auswertung von Fragebögen}
Wie werten wir die Fragenbögen aus?
