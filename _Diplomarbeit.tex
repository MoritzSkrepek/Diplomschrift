\documentclass[german,oneside,color]{htldipl}
% Zulässige Class Options: 
%   Hauptsprache: german (default), english
%   Doppelseitig: oneside (default), twoside
%   Syntax-Highlighting: color (default), black

% die folgende Zeile einkommentieren für Arial-Ähnliche Schriftart
%\renewcommand{\familydefault}{\sfdefault}

\graphicspath{{images/}}    % Bilderverzeichnis


\include{Settings}

\makeglossaries
\loadglsentries{glossary}					%beinhaltet Daten für das Glossar
\addbibresource{literatur.bib}     %beinhaltet Daten für das Literarturverzeichnis

%%%----------------------------------------------------------
\begin{document}
%%%----------------------------------------------------------
%Einstellungen an die eigene Diplomarbeit anpassen
\title{Applied Augmented Reality in Education}
\abteilung{Informaitk}
%\schwerpunkt{} wenn kein Ausbildungsschwerpunkt vorhanden ist z.B. Informatik
\schwerpunkt{Ausbildungsschwerpunkt Automatisierungstechnik}
\studienort{Wiener Neustadt}
\schule{HTBLuVA Wiener Neustadt}
\schullogo{htl.jpeg}
\abgabejahr{2034/24}
\betreuerA{Mag.\ BEd.\ Reis Markus}
\betreuerB{}
\betreuerC{}
%\betreuerD{} leer lassen wenn nicht vorhanden
\schuelerA{Moritz SKREPEK}
\evidenzA{5CHIF}
\subthemaA{Recherche zu Varianten von Knapsack-Algorithmen u. Umsetzung d. Knapsack-Problems als AR-Anwendungsszenario inkl. Dokumentation || Erstellen/Auswerten eines Feedbackfragebogens zur Lernunterstützung}
\schuelerB{Dustin LAMPEL}
\evidenzB{5CHIF}
\subthemaB{Design und Umsetzung der 3D-Objekte zur AR-Abbildung || Analyse der Steuerungsmöglichkeiten (Menüführung, Gesten, ...) und Erstellen der Benutzeroberfläche für die AR-Applikation mit Fokus auf UX}
\schuelerC{Seref HAYLAZ}
\evidenzC{5CHIF}
\subthemaC{Erfassen realer Objekte u. kontextgerechte Überlagerung d. Realität mit AR-Device || Tagging v. realen Elementen mittels QR-Codes für Tracking || Unit-Tests für d. implementierten Knapsack-Algorithmus}
\schuelerD{Jonas SCHODITSCH}
\evidenzD{5CHIF}
\subthemaD{Evaluierung/Auswahl Laufzeit-/Entwicklungsumgebung f. Umsetzung der Applikation u. Integration mit AR-Device inkl. Recherche || Konzeption/Umsetzung d. Anwendungsszenarios im Bereich Netzwerktechnik}
%\schuelerE{Elfriede NURNBERG-ATTACH}
%\evidenzE{5BHMIA-20}
%\subthemaE{Subthema E}
%\schuelerE{} leer lassen wenn nicht vorhanden
%\evidenzE{}
%\subthemaE{}



%%%----------------------------------------------------------
\frontmatter
\maketitle
\tableofcontents
%%%----------------------------------------------------------

\chapter{Vorwort}

Die vorliegende Diplomarbeit wurde im Zuge der Reife- und Diplomprüfung im Schuljahr 2023/24 an der Höheren Technischen
Bundeslehr- und Versuchsanstalt Wiener Neustadt verfasst. Die Grundlegende Idee zu dem arbeiten mit der Microsoft HoloLens2,
gefördert durch das Förderungsprogramm des Land Niederösterreich "Wissenschaft trifft Schule", lieferte uns unser Betreuer
Mag. BEd. Markus Reis. Das Ergebniss dieser Diplomarbeit ist eine Augmented Reality Applikation für die Verwendung innerhalb
des Unterrichts und am Tag der offenen Tür.

Besonderer Dank gebührt unserem Betruer Mag. Markus Reis  für sein unerschöpfliches Engagement und seine kompetente
Unterstützung. Weiteres möchten wir uns bei unserem Abteilungsvorstand Mag. Nadja Trauner sowie unserem Jahrgangsvorstand
MSc. Wolgang Schermann bedanken, die uns die gesamte Zeit an dieser Schule unterstützt haben.
				%ggfs. weglassen
\include{dokumentation}
\chapter{Kurzfassung}

Diese Diplomschrift befasst sich mit der Konzeption einer Lernapplikation
für die HTL Wiener Neustadt, sowie der Realisierung in Form von 
einer augmented reality Applikation auf der Microsoft HoloLens2.

Das Produkt setzt sich aus dem Hauptmenu, dem Ping Level und dem
Knappsack-Problem Level in Form eines Unreal Engine 5 Programms zusammen.

In der Applikation können die Schüler am Tag der offenen Tür zwei wichtige
Grundprinzipien der Informatik mit Hilfe von Augmented Reality interessant und
spielerisch kennenlernen und dadurch erkennen, ob Sie sowas interessiert.
		
\chapter{Abstract}

\begin{english} %switch to English language rules
This diploma thesis focuses on the development of an educational application for HTL Wiener Neustadt using the Unity
platform. The implementation took the form of an augmented reality (AR) application specifically designed for the
Microsoft HoloLens 2.

The application comprises three distinct levels, namely the main menu, the Ping Level, and the Knapsack-Problem
Level, all implemented using Unity.

During the open house event, the application enables students to explore two fundamental principles of computer
science in a playful and engaging manner through augmented reality. This provides students with the opportunity
to discover whether they have an interest in such topics. The utilization of Unity as the development platform
facilitated a comprehensive and scientifically grounded realization of this project.
\end{english}


			

%%%----------------------------------------------------------
\mainmatter           %Hauptteil (ab hier arab. Seitenzahlen)
%%%----------------------------------------------------------

\chapter{Einleitung}
\label{cha:Einleitung}

\section{Ausgangslage}\marginpar{\small\(\rightarrow\) {\tiny SKREPEK}}
Um dem IT-Fachkräftemangel entgegenzuwirken, ist es entscheidend, die Ausbildung im MINT-Bereich attraktiver zu gestalten.
Diese Diplomarbeit zielt darauf ab, einen wichtigen Beitrag zu diesem Ziel zu leisten, unterstützt durch das Förderprogramm
\textit{Wissenschaft trifft Schule} des Landes Niederösterreich. Dabei sollen exemplarische Anwendungen im Bereich Augmented
Reality für die Vermittlung von Informatik-Lehrinhalten evaluiert und umgesetzt werden.

\section{Auslöser}
Die Besucher des \textit{Tag der offenen Tür} erhalten durch diese Applikation eine Vorführung der neuesten Technologien,
was dazu beiträgt, dass sie erkennen, dass die Schule einen sehr hohen Technologiestandard aufweist. Dies soll sicherstellen,
dass sich auch in Zukunft viele interessierte Schüler für eine Ausbildung an der Abteilung für Informatik entscheiden. Darüber
hinaus stärkt dies den Ruf der Schule nach außen und präsentiert sie als attraktiven Ausbildungsstandort für zukünftige Mitarbeiter
vieler Unternehmen.

\section{Aufgabenstellung}
Die Aufgabenstellung besteht darin, zwei Anwendungszenarien zu erstellen, die jeweils ein Grundprinzip der Informatik
veranschaulichen. Im ersten Szenario wird das Grundprinzip des Nachrichtenaustauschs zwischen zwei PCs dargestellt. Das Ziel
hierbei ist es dem Benutzer das eigentlich unsichtbare Nachrichtenpaket zu visualisieren. Der Benutzer soll hier auf einer
Website eine Nachricht eingeben können, um diese dann an den anderen Computer zu senden und zusätzlich kann auch der Inhalt
dieses Pakets durch draufdrücken visualisiert werden.

Im zweiten Szenario wird das in der IT bekannte und vielseitige verwendete Optimierungsproblem des Knapsack-Problems dargestellt.
Mit Hilfe von Augmented Reality wird der Rucksack als Inventar angezeigt in welchem der Benutzer alltägliche Gegenstände
platzieren kann. Die Aufgabe des Benutzers besteht darin, den Rucksack bestmöglich mit den verfügbaren Gegenständen
zu füllen, um den Gesamtwert den Rucksacks zu maximieren. Zusätzlich zur Lösung des Problems kann der Benutzer durch
einen Knopfdruck eine der berechneten perfekten Lösungen anzeigen.

\section{Team}
Das Diplomarbeitsteam besteht aus:
\begin{itemize}
    \item Moritz SKREPEK
    \item Seref HAYLAZ
    \item Dustin LAMPEL
    \item Jonas SCHODITSCH
\end{itemize}

\subsection{Aufteilung}
Die Rolle des Projektsleiters der Diplomarbeit nahm Moritz SKREPEK ein, da dieser die grundlegende Idee für die Darstellung
zweier Anwendungsszenarios mittels der Microsoft HoloLens2 hatte.

Die entwickelte Applikation lässt sich in das Hauptmenü, das Nachrichtenaustausch-Szenario und das Knapsack-Problem-Szenario
gliedern. Die Implementierung sowie die gesamte UI/UX und das Frontend der Website übernahm Dustin LAMPEL unter Verwendung
des UX-Tools-Plugins, welches sämtliche UI/UX Elemente für Mixed Reality Anwendungen bereitstellt und HTML, CSS und JavaScript.
Die Umsetzung des Nachrichtenaustausch-Szenarios übernahm SCHODITSCH Jonas mittels Objekt-Tracking und Foto-Aufnahmen. Für
die Implementierung des Knapsack-Problem-Szenarios waren Moritz SKREPEK und Seref HAYLAZ mittels Verwendung von Plane-detection,
QR-Code Tagging und Tracking, Knapsack Algorithmus und Unittests zuständig.




\include{diplomschrift}
\include{latex}
\include{faq}
\include{abbildungen}
\include{mathematik}
\chapter{Literatur}

\begin{sloppypar}
    Blender. URL: \url{https://www.blender.org/about/}
    (besucht am 06.10.2023).
\end{sloppypar}

\begin{sloppypar}
    Scrum Alliance Inc.
    WHAT IS SCRUM? URL: \url{https://www.scrumalliance.org/about-scrum#}
    (besucht am 06.10.2023).
\end{sloppypar}

\begin{sloppypar}
    Epic Games, Unreal Engine.
    URL: \url{https://www.unrealengine.com/marketplace/en-US/product/mixed-reality-toolkit-hub}
    (besucht am 06.10.2023).
\end{sloppypar}

\begin{sloppypar}
    Epic Games, Unreal Engine.
    URL: \url{https://www.unrealengine.com/marketplace/en-US/product/mixed-reality-ux-tools}
    (besucht am 06.10.2023).
\end{sloppypar}

\begin{sloppypar}
    Epic Games, Unreal Engine.
    URL: \url{https://www.unrealengine.com/marketplace/en-US/product/ef8930ca860148c498b46887da196239}
    (besucht am 06.10.2023).
\end{sloppypar}

\begin{sloppypar}
    Epic Games, Unreal Engine.
    URL: \url{https://docs.unrealengine.com/4.27/en-US/ProgrammingAndScripting/Blueprints/UserGuide/Types/}
    (besucht am 06.10.2023).
\end{sloppypar}

\begin{sloppypar}
    Epic Games, Unreal Engine.
    URL: \url{https://docs.unrealengine.com/4.27/en-US/ProgrammingAndScripting/Blueprints/}
    (besucht am 06.10.2023).
\end{sloppypar}


\begin{sloppypar}
    Scholl, Armin. \textit{Die Befragung} 3. Aufl. Stuttgart: utb GmbH, 2014.
\end{sloppypar}

\begin{sloppypar}
    Mayer, Horst. \textit{Interview und schriftliche Befragung. Entwicklung, Durchführung und Auswertung} 3. Aufl. München: R. Oldenbourg, 2008.
\end{sloppypar}

\begin{sloppypar}
    Bühner, Markus. \textit{Einführung in die Test- und Fragebogenkonstruktion.} 3. Aufl. München: Pearson Studium, 2021.
\end{sloppypar}
\include{drucken}
\include{word}
\include{schluss}

%%%----------------------------------------------------------
%%%Anhang
\appendix
%\chapter{Mockups}

\section{Hauptmenu}
\begin{figure}[H]
    \centering
    \begin{minipage}[c][\textheight][c]{\textwidth}
        \centering
        \includegraphics[scale=0.7]{images/HauptmenuMockup}
    \end{minipage}
\end{figure}
\newpage

\section{Nachrichtenaustausch Anwendungsszenario}
\begin{figure}[H]
    \centering
    \begin{minipage}[c][\textheight][c]{\textwidth}
        \centering
        \includegraphics[scale=0.75]{images/NachrichtenaustauschMockup}
    \end{minipage}
\end{figure}
\newpage

\section{Knappsack-Problem Anwendungsszenario}
\begin{figure}[H]
    \centering
    \begin{minipage}[c][\textheight][c]{\textwidth}
        \centering
        \includegraphics[scale=0.75]{images/MockUpKnapsack}
    \end{minipage}
\end{figure}
\newpage

	% Technische Ergänzungen
\chapter{Abläufe}

\section{Hauptmenu}
\begin{figure}[htbp]
	\centering
	\includegraphics[width=0.80\textwidth]{images/Hauptmenu_Ablauf}
	\caption{Hauptmenu UML Aktivitätsdiagramm}
\end{figure}
\newpage

\section{Nachrichtenaustausch Anwendungsszenario}
\begin{figure}[htbp]
	\centering
	\includegraphics[width=\textwidth]{Nachrichtenaustausch_Ablauf.png}
	\caption{Nachrichtenaustausch UML Aktivitätsdiagramm}
\end{figure}
\newpage

\section{Knapsack-Problem Anwendungsszenario}
\begin{figure}[htbp]
	\centering
	\includegraphics[width=\textwidth]{images/Knapsack_Ablauf.png}
	\caption{Knapsack-Problem UML Aktivitätsdiagramm}
\end{figure}
\newpage	% Inhalt der CD-ROM/DVD
\include{anhang_c}	% Chronologische Liste der Änderungen
\include{anhang_d}	% Quelltext dieses Dokuments

%%%----------------------------------------------------------
%Ausgabe der automatischen Zusatzdaten: Glossar, Index, Literaturverzeichnis
\clearpage
\printglossaries

\clearpage
\chapter*{Index}
\addcontentsline{toc}{chapter}{Index}
\printindex[allgemein]

\printindex

\printindex[name]

\printindex[title]


%Literaturverzeichnis
\clearpage
\addcontentsline{toc}{chapter}{\bibname}

\printbibliography


%%%----------------------------------------------------------

%%%Messbox zur Druckkontrolle
%\include{messbox}

\end{document}
