\chapter{Produktspezifikationen}
Dieses Kapitel behandelt die Planung und Spezifikation des Projekts.
Weiteres wird die verwendete Technologieauswahl begründet und mit Alternativlösungen verglichen.

\section{Anforderungen und Spezifikationen}
Hier steht der allgemeine Text für die Anforderungen und Spezifikationen

\subsection{Use Cases}
Hier steht der allgemeine Text für die Use Cases

\section{Design}
Hier steht der allgemeine Text für das Design

\subsection{Abläufe}
Hier steht der allgemeine Text für die Abläufe

\subsection{Mockups}
Hier steht der allgemeine Text für die Mockups

\section{Eingesetzte Technologien}

\subsection{Kriterien}
Bei der Auswahl der eingesetzten Technologien war es besonders wichtig, dass diese möglichst
zuverlässig und bereits etabliert sind. Die Technologien sollen ausfallsicher, leicht benutzbar
und vorallem eine performant Verwendung der Applikation sicherstellen.

\subsection{Game Engine}
Als Game Engine wird eine Entwicklungsumgebung für das Design und Entwickeln von Spielen bezeichnet.
Zu Projektstart gab es die Auswahl zwischen den zwei bekanntesten Game Enginges, die momentan am Markt
vorhanden sind. Diese sind die folgenden:
\begin{itemize}
    \item Unreal Engine
    \item Unity
\end{itemize}

Nach tiefgründiger Recherche war für das Projektteam klar, dass Unity die verwendete GameEngine sein wird.
Folgende Kriterien haben uns in dieser Entscheidung verstärkt:
\begin{itemize}
    \item Programmiersprache: C#.
    \item Einfacher Einstieg in die Entwicklung von Spielen für Beginner.
    \item Sehr gute Dokumentation.
    \item Hohe Anzahl an Tutorials an die man sich richten kann.
\end{itemize}

\subsection{Unity foundation packages}
%Quellen:
In dem folgenden Abscnhitt wird erklärt welche Packages in die Unity Applikation
eingeführt werden müssen um die Entwicklung einer Augmented Reality Applikation ohne Problem
ermöglichen zu können.

\subsubsection{MRTK3}
Das Mixed Reality Toolkit (MRTK) \footnote{Microsoft \cite{MRTK3}} ist eine Sammlung von Tools,
Skripten und Ressourcen, die speziell für die Entwicklung von Mixed-Reality-Anwendungen, einschließlich Augmented
Reality, in Unity entwickelt wurden. MRTK3 ist eine Weiterentwicklung der vorherigen Versionen und bietet viele
Vorteile für AR-Anwendungen:
\begin{itemize}
    \item Interaktions- und Benutzerführung: \\
    MRTK3 stellt eine Reihe von Interaktionskomponenten und -systemen zur
    Verfügung, die es Entwicklern ermöglichen, intuitivere Benutzererfahrungen in AR-Anwendungen zu gestalten.
    Dies umfasst Dinge wie das Platzieren von Objekten in der realen Welt, die Verfolgung von Handgesten und die
    Unterstützung von Blickverfolgung.
    \item Standardisierte APIs: \\
    Durch die Verwendung von MRTK3 kannst du auf standardisierte APIs und Komponenten
    zugreifen, die speziell für AR-Anwendungen entwickelt wurden. Dies erleichtert die Implementierung von Funktionen
    wie Handgesten, Sprachsteuerung und Objektplatzierung.
    \item Einfache Konfiguration und Anpassung: \\
    MRTK3 bietet eine einfache Konfiguration und Anpassung über die
    Unity-Oberfläche. Dies erleichtert die Anpassung deiner AR-Anwendung an spezifische Anforderungen und Use Cases.
\end{itemize}

\subsubsection{Microsoft OpenXR Plugin}
Das Microsoft OpenXR Plugin \footnote{Khronos \cite{OpenXR}} ist eine Sammlung von Tools ist ein wichtiges Plugin für Unity, das die Integration von OpenXR-Unterstützung in
die AR-Anwendung ermöglicht. OpenXR ist ein offener Industriestandard, der die Entwicklung von XR
(Extended Reality)-Anwendungen, einschließlich Augmented Reality, erleichtert. Anschließend ein paar Punkte wieso
dieses Plugin so wichtig ist:
\begin{itemize}
    \item Geräteunabhängigkeit: \\
    Durch die Verwendung von OpenXR und dem Microsoft OpenXR Plugin kann die AR-Anwendung auf verschiedenen
    XR-Geräten ausgeführt werden, ohne die Kernfunktionalität für jedes einzelne Gerät neu entwickeln zu müssen.
    Dies gewährleistet eine reibungslose Interaktion mit der HoloLens 2 und anderen XR-Geräten.
    \item Leistungssteigerung und Stabilität: \\
    Die Nutzung von OpenXR und des Microsoft OpenXR Plugins kann die
    Leistung und Stabilität der AR-Anwendung erheblich verbessern. Sie gewährleisten eine reibungslose Ausführung
    der Anwendung auf dem Zielsystem und bieten eine optimale Benutzererfahrung.
\end[itemize]

\subsection{Rendering Program}
Das Rendering Program wird benötigt um die eingesetzten 3D-Modellen für die zwei
Level zu erstellen. Die Auswahl dieses Rendering Programs Blender war bereits bei Projektstart
klar. \\
Diese Entscheidung ist begründet durch folgende Punkte:
\begin{itemize}
    \item Kostenfrei und Opensource\\
    Blender ist kostenfrei und quelloffen, was bedeutet, dass Sie es ohne Lizenzkosten nutzen
    können. Das kann bei der Entwicklung von AR-Anwendungen mit begrenztem Budget besonders
    attraktiv sein.
    \item Echtzeit-Rendering\\
    Blender verfügt über einen Echtzeit-Renderer namens Eevee, der schnelle Vorschauen und
    Renderings ermöglicht. Dies kann nützlich sein, um AR-Inhalte in Echtzeit anzuzeigen und zu
    überprüfen.
    \item Integration mit AR-Frameworks\\
    Obwohl Blender nicht direkt AR-Funktionen unterstützt, können Sie die erstellten 3D-Modelle
    und Animationen in AR-Entwicklungsumgebungen wie Unity oder Unreal Engine importieren und
    dort AR-spezifische Funktionalitäten hinzufügen.

\end{itemize}
Wie funktioniert Blender im Allgemeinen? Die folgende Beschreibung hebt die Schlüsselaspekte, sowie die Funktionalität von Blender hervor.
\begin{itemize}
	\item Benutzeroberfläche und Interaktion\\
	Blender verfügt über eine komplex gestaltete Benutzeroberfläche, die sich jedoch durch eine hohe Anpassbarkeit auszeichnet. Die Hauptansicht enthält 3D-Modelle, Ansichten, Fenster und Panels. Benutzer interagieren mit den Objekten und Werkzeugen über Maus- und Tastaturbefehle. Erfahrene Nutzer sind in der Lage einige Zeit zu sparen, in dem sie sogenannte Hotkeys oder Shortcuts verwenden. Wenn man zum Beispiel die Tasten "STRG" und "A" gleichzeitig drückt, kann man alle Objekte auswählen, ohne diese mit der Maus einzeln anzuklicken und auszuwählen.
	\item 3D-Modellierung\\
	Blender ermöglicht die Erstellung von 3D-Modellen mithilfe von Primitiven wie Würfeln, Kugeln, Flächen und Kurven. Diese können dann bearbeitet und modifiziert werden, um komplexe Formen zu erstellen. Zu den Modellierungswerkzeugen gehören Extrusion, Verschiebung, Skalierung und Rotation.
	\item Animation\\
	Blender unterstützt die Erstellung von Animationen durch das Setzen von Schlüsselbildern (Keyframes) und die Interpolation von Position, Rotation und Skalierung zwischen diesen Schlüsselbildern. Es bietet auch fortschrittliche Animationstools für Skelettanimation, Constraints und Pfadbewegungen.
	\item Materialien und Texturen\\
	Für die Erzeugung realistischer Oberflächen können Materialien erstellt und Texturen auf Objekte angewendet werden. Blender ermöglicht die Feinanpassung von Materialeigenschaften wie Diffusreflexion, Glanz, Transparenz und Emission.
	\item Rendering\\
	Blender verfügt über einen integrierten Renderer, der hochwertige Bilder und Animationen erstellen kann. Benutzer können Render-Einstellungen anpassen und Bilder in verschiedenen Formaten exportieren.
	\item Skripting und Add-Ons\\
	Fortgeschrittene Benutzer können Blender durch Python-Skripting anpassen und erweitern, um benutzerdefinierte Tools und Automatisierungen zu erstellen. Darüber hinaus gibt es eine aktive Entwicklergemeinschaft, die Add-Ons entwickelt, um die Funktionalität von Blender zu erweitern.
	\item Gemeinschaft und Ressourcen\\
	Blender verfügt über eine engagierte Benutzergemeinschaft, die eine umfangreiche Dokumentation, Tutorials und Foren bereitstellt. Diese Ressourcen erleichtern die Einarbeitung und die Lösung von Problemen.
\end{itemize}
In unserer Diplomarbeit kommt Blender in beiden Leveln zum Einsatz. Die Hauptanwendung des Programms findet in Level 2 statt. Dabei wird Blender für die digitale Modellierung der wichtigen täglichen Gegenstände von Schülern eingesetzt. Das Ziel am Ende ist es, ein großes Pool aus einigen Objekten zu haben, um den Benutzern eine gute Auswahl zu geben.
