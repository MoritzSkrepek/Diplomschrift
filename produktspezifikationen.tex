\chapter{Produktspezifikationen}
Dieses Kapitel behandelt die Planung und Spezifikation des Projekts.
Weiteres wird die verwendete Technologieauswahl begründet und mit Alternativlösungen verglichen.

\section{Anforderungen und Spezifikationen}
Hier steht der allgemeine Text für die Anforderungen und Spezifikationen

\subsection{Use Cases}
Hier steht der allgemeine Text für die Use Cases

\section{Design}
Hier steht der allgemeine Text für das Design

\subsection{Abläufe}
Hier steht der allgemeine Text für die Abläufe

\subsection{Mockups}
Hier steht der allgemeine Text für die Mockups

\section{Eingesetzte Technologien}

\subsection{Kriterien}
Bei der Auswahl der eingesetzten Technologien war es besonders wichtig, dass diese möglichst
zuverlässig und bereits etabliert sind. Die Technologien sollen ausfallsicher, leicht benutzbar
und vorallem eine performant Verwendung der Applikation sicherstellen.

\subsection{Game Engine}
Als Game Engine wird eine Entwicklungsumgebung für die Erstellung von Spielen bezeichnet.
Hier gab es am Projektanfang die Auswahl zwischen zwei bekannten Game Engines.
\begin{itemize}
    \item Unreal Engine
    \item Unity
\end{itemize}
Ausschlaggebend für die Entscheidung von Unreal Engine waren mehrere Punkte. Einersets ist es
die leistungsstarke Grafikleistung die Unreal Engine unterstützt und Unity weit überlegen ist.
Ein weiterer Punkt ist das Blueprint-Scripting-System, dass einen leichten und schnellen
Einstieg in die Entwicklung einer AR-Applikation ermöglicht.

\subsection{Plugins}
%Quellen:
%https://www.unrealengine.com/marketplace/en-US/product/mixed-reality-toolkit-hub
%https://www.unrealengine.com/marketplace/en-US/product/mixed-reality-ux-tools
%https://www.unrealengine.com/marketplace/en-US/product/ef8930ca860148c498b46887da196239
In dem folgenden Abschnitt wird erklärt welche Plugins in dem Unreal Editor
Installiert und verwendet werden müssen, dass die Entwicklung einer Augmented Reality
Applikation in dem Unreal Editor möglich ist.

\subsubsection{Microsoft OpenXR}
Bei Absicht eine Applikation für die HoloLens2 oder ein Windows Moexed Reality VR-Headset
zu schreiben, ist dieses Plugin notwendig. Dieses Plugin enthält eine Reihe von
OpenXR-Erweiterungen, die wichtige Mixed-Reality-spezifische Funktionen von Microsoft freischaltet. \\
Darunter sind folgende für diese Diplomarbeit wichtige Funktionen eingeschlossen:
\begin{itemize}
    \item Spatial Mapping
    \item Spatial Anchors
    \item Holographic remoting
    \item QR-Tracking
\end{itemize}

\subsubsection{Mixed Reality Toolkit Hub}
Das MRTK Hub Plugin ist eine Komponente des Mixed Reality Toolkits, die dazu dient, die
Interaktion zwischen Benutzern und MR/AR-Anwendungen zu erleichtern. Es bietet eine Reihe von
Tools und Funktionen, um Benutzererfahrungen in MR- und AR-Anwendungen zu verbessern. Hier
sind einige wichtige Aspekte des Plugins:
\begin{itemize}
    \item Benutzeroberfläche (UI) und Menüs\\
    Das MRTK Hub Plugin ermöglicht die einfache Erstellung von Benutzeroberflächen und Menüs,
    die in der AR-Anwendung angezeigt werden können. Diese können verwendet werden, um Optionen,
    Steuerelemente und Informationen für den Benutzer bereitzustellen.
    \item Benutzererlebnis-Design\\
    Das Plugin unterstützt die Gestaltung von Benutzererfahrungen (UX) in AR-Anwendungen,
    einschließlich der Platzierung von virtuellen Objekten, der Navigation und der
    Benutzeroberfläche.
    \item Interaktion und Eingabe\\
    Das Plugin bietet verschiedene Funktionen zur Unterstützung von Eingabe- und
    Interaktionsmöglichkeiten in AR-Umgebungen. Dazu gehören Hand- und Gestenerkennung sowie
    die Integration von Controllern und Tracking-Systemen.
\end{itemize}

\subsubsection{Mixed Reality UX Tools}
Das Mixed Reality UX Tools Plugin bietet eine Reihe von Funktionen und Werkzeugen, die es
Entwicklern ermöglichen, immersive und benutzerfreundliche Mixed-Reality-Erfahrungen in der
Unreal Engine zu erstellen. Es ist darauf ausgerichtet, die Entwicklungsarbeit für AR- und
MR-Anwendungen zu erleichtern und die Interaktion zwischen der digitalen und der physischen
Welt zu optimieren.
\begin{itemize}
    \item Interaktionstools\\
    Das Mixed Reality UX Tools Plugin enthält Werkzeuge zur Implementierung von Interaktionen
    in AR-Anwendungen. Dies kann die Erkennung von Gesten, Handbewegungen und Touch-Eingaben
    umfassen, um Benutzern die Interaktion mit virtuellen Objekten zu ermöglichen.
    \item Tracking und Kalibrierung\\
    Das Plugin bietet Möglichkeiten zur Tracking-Optimierung und zur Kalibrierung von
    AR-Geräten. Dies hilft dabei, genaue AR-Positionierung und -Orientierung sicherzustellen,
    um realistische AR-Inhalte zu erzeugen.
    \item Benutzeroberflächen-Design\\
    Das Mixed Reality UX Tools Plugin kann auch Werkzeuge zur Gestaltung von Benutzeroberflächen
    (UI) in AR-Anwendungen enthalten. Dies ermöglicht die Integration von AR-spezifischen
    Benutzerschnittstellen, um Informationen und Steuerelemente in die AR-Erfahrung einzuführen.
\end{itemize}

\subsection{Rendering Program}
%https://www.blender.org/about/
Das Rendering Program wird benötigt um die eingesetzten 3D-Modellen für die zwei
Level zu erstellen. Die Auswahl dieses Rendering Programs Blender war bereits bei Projektstart
klar. \\
Diese Entscheidung ist begründet durch folgende Punkte:
\begin{itemize}
    \item Kostenfrei und Opensource\\
    Blender ist kostenfrei und quelloffen, was bedeutet, dass Sie es ohne Lizenzkosten nutzen
    können. Das kann bei der Entwicklung von AR-Anwendungen mit begrenztem Budget besonders
    attraktiv sein.
    \item Echtzeit-Rendering\\
    Blender verfügt über einen Echtzeit-Renderer namens Eevee, der schnelle Vorschauen und
    Renderings ermöglicht. Dies kann nützlich sein, um AR-Inhalte in Echtzeit anzuzeigen und zu
    überprüfen.
    \item Integration mit AR-Frameworks\\
    Obwohl Blender nicht direkt AR-Funktionen unterstützt, können Sie die erstellten 3D-Modelle
    und Animationen in AR-Entwicklungsumgebungen wie Unity oder Unreal Engine importieren und
    dort AR-spezifische Funktionalitäten hinzufügen.
\end{itemize}
