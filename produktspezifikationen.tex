\chapter{Produktspezifikationen}
Dieses Kapitel behandelt die Planung und Spezifikation des Projekts.
Weiteres wird die verwendete Technologieauswahl begründet und mit Alternativlösungen verglichen.

\section{Eingesetzte Technologien}

\subsection{Microsoft HoloLens2}\marginpar{\small\(\rightarrow\) {\tiny SKREPEK}}
Zentrale Plattform des Projekts ist die Augmented-Reality-Brille "HoloLens 2" von Microsoft. Dabei handelt es sich um
eine bahnbrechende Augmented-Reality-Brille, die eine revolutionäre Verbindung zwischen der digitalen und der physischen
Welt schafft. Mit einer Reihe hochentwickelter Sensoren, Kameras und einem hochauflösenden Display eröffnet die HoloLens
2 neue Horizonte für interaktive und immersive AR-Erlebnisse.

Die HoloLens 2 bietet eine beeindruckende Immersion, die es dem Nutzer ermöglicht, digitale Objekte nahtlos in seine
Umgebung zu integrieren. Dank des fortschrittlichen Hand- und Blickverfolgungssystems können Benutzer mit den holografischen
Elementen interagieren, als wären sie Teil ihrer realen Welt. Dies ermöglicht eine breite Palette von Anwendungen, von
Unterhaltung über Bildung bis hin zu industriellen Anwendungen.

\subsection{Kriterien}
Bei der Auswahl der Technologien, die für die Entwicklung der geplanten Anwendung eingesetzt werden, war es besonders
wichtig, dass diese über eine gute Dokumentation und eine große Entwicklergemeinschaft verfügen, falls Fragen auftauchen,
für die nicht direkt im Internet eine Lösung gefunden werden kann. Außerdem sollten sie einfach zu bedienen sein und vor
allem eine performante Nutzung der Anwendung gewährleisten.

\subsection{Game Engine}
Um einen reibungslosen Verlauf des Projekts zu gewährleisten, ist die sorgfältige Auswahl der richtigen \textit{Game Engine} von
entscheidender Bedeutung. Die Game Engine fungiert als fundamentale Plattform für die Entwicklung und Erstellung von
Videospielen, indem sie eine umfassende Palette von Werkzeugen, Bibliotheken und Funktionen bereitstellt, die Entwicklern
hilft, Spiele zu konzipieren, umzusetzen und zu optimieren. In diesem Abschnitt werden zwei potenzielle Game Engines,
die für das Projekt in Betracht gezogen wurden, eingehend untersucht und anschließend die Gründe für die getroffene Auswahl
erläutert.

\subsubsection{Unity}
Unity ist eine Game Engine, die von Entwicklern weltweit für die Erstellung von 2D- und 3D-Spielen genutzt wird. Sie
unterstützt verschiedene Plattformen wie PC, Konsolen, Mobilgeräte und AR/VR-Geräte\footnote{Unity.com \cite{Plattformen}}. Unity bietet Entwicklern eine
umfangreiche Sammlung von Werkzeugen und Ressourcen, um Spiele schnell zu prototypen und zu entwickeln.

Ein herausragendes Merkmal von Unity ist der Asset Store. Hier können Entwickler Assets wie 3D-Modelle, Texturen, Sounds
und Plugins kaufen oder verkaufen. Dadurch können sie ihre Projekte mit hochwertigen Inhalten erweitern und verbessern,
ohne alles von Grund auf neu erstellen zu müssen.\footnote{Assetstore.Unity.com \cite{Unity Asset Store}} Unity bietet außerdem eine starke Community-Unterstützung mit Foren,
Tutorials und Schulungen, was besonders für neue Entwickler hilfreich ist.

Die Programmierung in Unity erfolgt hauptsächlich durch die Verwendung von C-Sharp. Diese ist sowohl für
erfahrene Entwickler als auch für Anfänger zugänglich. Unity ist aufgrund der Kombination von benutzerfreundlichen Werkzeugen,
einer großen Community und einer breiten Plattformunterstützung eine beliebte Wahl für Indie-Entwickler sowie für große Studios.

\subsubsection{Unreal Engine}
Die Unreal Engine ist eine leistungsstarke Game Engine, die für ihre hochwertigen Grafiken und fortgeschrittenen Funktionen
bekannt ist. Sie wird häufig für die Entwicklung von AAA-Titeln sowie für hochwertige VR-Erfahrungen verwendet. Die Engine
bietet branchenführende Grafikfunktionen wie fortschrittliche Beleuchtung, Partikelsysteme, Physiksimulationen und Echtzeit-Rendering.

Eine bemerkenswerte Funktion der Unreal Engine ist das Blueprints-System (siehe folgender Absatz). Es ermöglicht Entwicklern, Spiele und interaktive
Inhalte ohne herkömmlichen Programmiercode zu erstellen. Dadurch wird die Engine besonders zugänglich für Künstler und
Designer, die möglicherweise keine tiefen Programmierkenntnisse haben.

Die Unreal Engine bietet eine umfangreiche Sammlung von vorgefertigten Assets und Werkzeugen sowie einen integrierten
Marketplace, auf dem Entwickler zusätzliche Inhalte erwerben können. Sie unterstützt eine Vielzahl von Plattformen,
darunter PC, Konsolen, Mobilgeräte und VR-Headsets.

Insgesamt ist die Unreal Engine eine leistungsstarke Wahl für Entwickler, die hochwertige Grafiken und fortschrittliche
Funktionen in ihren Spielen und Anwendungen benötigen. Sie wird häufig von größeren Studios genutzt, die Zugang zu hochwertigen
Tools und Support benötigen.\footnote{Unrealengine.com \cite{Wir liefern die Engine. Sie machen sie Unreal}}

\subsubsection*{Blueprint Visual Scripting}
Das Blueprint Scripting System ist eine leistungsstarke visuelle Programmierumgebung innerhalb der Unreal Engine. Es
ermöglicht Entwicklern, komplexe Logik und Interaktionen ohne traditionelle Programmierkenntnisse zu erstellen. Das System
basiert auf dem Konzept von \textit{Blueprints}, die visuelle Darstellungen von Logik und Funktionen sind.

Die Verwendung von Blueprints bietet eine Reihe von Vorteilen. Einerseits ermöglicht es Künstlern und Designern ohne
tiefgreifende Programmiererfahrung, komplexe Interaktionen und Spielmechaniken zu implementieren. Durch das Drag-and-Drop-Interface
können Benutzer Funktionsblöcke miteinander verbinden und so den Fluss der Spiellogik steuern.

Andererseits erleichtert das Blueprint-System die Iteration und Prototypisierung. Da Änderungen visuell vorgenommen
werden können, können Entwickler schnell experimentieren und Anpassungen vornehmen, um das gewünschte Verhalten zu erreichen.
Dies beschleunigt den Entwicklungsprozess und ermöglicht es Teams, flexibel auf Feedback und sich ändernde Anforderungen zu reagieren.

Das Blueprint Scripting System bietet zudem eine hohe Flexibilität und Erweiterbarkeit. Entwickler können benutzerdefinierte
Blueprints erstellen und sie in anderen Projekten wiederverwenden, was die Effizienz erhöht und die Entwicklung
beschleunigt.\footnote{Unreal Engine Dokumentation \cite{Blueprints Visual Scripting}}

\subsubsection{Game Engine Auswahl und Wechsel im Projektverlauf}
Bei Projektbeginn wurde nach umfassender Recherche und Evaluierung verschiedener Optionen entschieden, die Unreal Engine
als primäre Entwicklungsumgebung für dieses Projekt zu verwenden. Diese Entscheidung wurde aufgrund mehrerer überzeugender
Faktoren getroffen, darunter insbesondere das beliebte Blueprint Visual Scripting, das die Entwicklung von interaktiven
Inhalten erleichtert und auch für Personen ohne umfangreiche Programmierkenntnisse zugänglich macht. Zusätzlich zu diesem
herausragenden Merkmal wurden weitere Aspekte berücksichtigt, die die Unreal Engine zu einer attraktiven Wahl machten,
wie beispielsweise ihre leistungsstarken Grafikfunktionen, die branchenweit anerkannt sind, sowie ihre Unterstützung für
die Entwicklung hochwertiger VR-Erfahrungen und AAA-Titel.

\subsubsection*{Herausforderungen im ersten Monat des Entwicklungsprozesses}
Trotz der zu Beginn getroffenen Entscheidung für die Unreal Engine traten im Verlauf des ersten Monats des Entwicklungsprozesses
bestimmte Herausforderungen auf. Diese Herausforderungen können als unerwartete Schwierigkeiten betrachtet werden, die während
der Umsetzung des Projekts auftraten und zweifel an der ursprünglichen Entscheidung mit sich brachte.
Im Einzelnen wurden folgende Herausforderungen identifiziert:
\begin{enumerate}
    \item \textbf{Mangelhafte Dokumentation für AR-Entwicklung in der Unreal Engine:} Die unzureichende Dokumentation
    für die Entwicklung von Augmented Reality (AR)-Anwendungen in der Unreal Engine erwies sich als erhebliche Hürde.
    Fehlende detaillierte Anleitungen und Referenzen für AR-spezifische Funktionen behinderten die effiziente Integration von AR-Elementen.
    \item \textbf{Begrenzte Verfügbarkeit von AR-spezifischen Online-Tutorials:} Ein Mangel an Online-Tutorials,
    die sich speziell mit der Entwicklung von AR-Anwendungen in der Unreal Engine befassten, führte zu einer
    beträchtlichen Lernkurve für das Entwicklerteam und verzögerte den Implementierungsprozess von AR-spezifischen Features.
    \item \textbf{Komplexität der AR-Entwicklung in der Unreal Engine:} Die Unreal Engine erwies sich als sehr
    anspruchsvoller für Neueinsteiger in Bezug auf die Umsetzung von AR-spezifischen Funktionen. Die Notwendigkeit, komplexe Skripte zu
    erstellen und vielfältige Einstellungen anzupassen, führte zu einem erhöhten Zeitaufwand für die Umsetzung von AR-Elementen.
    \item \textbf{Eingeschränkte Community-Unterstützung für AR-Entwicklung:} Im Vergleich zu Unity war die Community-Unterstützung
    für die AR-Entwicklung in der Unreal Engine begrenzt. Die Verfügbarkeit von Ratschlägen und Lösungen für spezifische
    AR-Herausforderungen war eingeschränkt, was die Eigenständigkeit bei der Lösung von Problemen beeinträchtigte.
\end{enumerate}

Diese Herausforderungen bildeten die Grundlage für die strategische Entscheidung des Projektteams, von Unreal Engine
zu Unity zu wechseln. Der Wechsel ermöglichte eine effizientere und zielführende Entwicklung der AR-Applikation,
gestützt durch Unity's umfassende Unterstützung, detaillierte Dokumentation und breite Community-Ressourcen.

\subsection{Unity foundation packages}
Um Augmented Reality (AR)-Applikationen in Unity erfolgreich zu entwickeln, sind bestimmte \textit{Foundation Packages}
erforderlich. Diese Pakete müssen in Unity importiert werden, um grundlegende Funktionalitäten bereitzustellen, die für
die erfolgreiche Umsetzung einer AR-Applikation notwendig sind. Im Folgenden werden die beiden wesentlichen Pakete näher
erläutert, die für die Funktionalität einer solchen Applikation unerlässlich sind.

\subsubsection{MRTK3}
Das Mixed Reality Toolkit 3 (MRTK3), welches Sowohl für Unity als auch in der Unreal Engine anwendbar ist, ist ein leistungsstarkes Open-Source-Entwicklungstoolkit, das von Microsoft entwickelt
und gepflegt wird. Es ist speziell darauf ausgerichtet, die Entwicklung von Mixed-Reality-Anwendungen zu erleichtern,
indem es eine umfassende Sammlung von Komponenten, Skripten und Assets bereitstellt. MRTK3 bietet Entwicklern eine Vielzahl
von Funktionen und Werkzeugen, die speziell für die Erstellung von Anwendungen für Augmented Reality (AR), Virtual Reality
(VR) und Mixed Reality (MR) optimiert sind.

Das Toolkit umfasst eine breite Palette von Funktionen, darunter Interaktionselemente wie Hand- und Gestenerfassung,
räumliches Mapping, Physiksimulationen für Objekte in der realen Welt, Benutzerschnittstellen-Design-Werkzeuge und vieles
mehr. MRTK3 ist plattformübergreifend und unterstützt verschiedene AR-/VR-Headsets sowie andere Mixed-Reality-Geräte.

Die Bedeutung von MRTK3 für die Entwicklung von AR-Applikationen liegt in seiner Fähigkeit, Entwicklern eine solide
Grundlage und eine Vielzahl von vordefinierten Komponenten und Tools zu bieten, die den Entwicklungsprozess beschleunigen
und vereinfachen. Durch die Verwendung von MRTK3 können Entwickler komplexe AR-Anwendungen schneller prototypisieren und
implementieren, da sie auf eine umfangreiche Bibliothek von Funktionen zugreifen können, die speziell für AR-Szenarien
optimiert sind. Dies erhöht die Effizienz der Entwicklung und ermöglicht es den Entwicklern, sich auf die Gestaltung und
Umsetzung innovativer AR-Erfahrungen zu konzentrieren, ohne sich um die Grundlagen der AR-Entwicklung kümmern zu müssen.\footnote{Microsoft-Dokumentation \cite{Mixed Reality Toolkit 3}}

\subsubsection{Microsoft OpenXR Plugin}
Das Microsoft OpenXR Plugin stellt eine bedeutende Komponente im Kontext der Entwicklung von Augmented Reality (AR)-Applikationen
innerhalb der Entwicklungsumgebung dar. Entwickelt und bereitgestellt von Microsoft, ermöglicht dieses Plugin die
nahtlose Integration des OpenXR-Standards in Unity. OpenXR, initiiert durch die Khronos Group, fungiert als branchenweiter
Standard zur Vereinheitlichung der XR-Anwendungsentwicklung, wobei XR für Extended Reality steht und sowohl Augmented
Reality (AR), Virtual Reality (VR) als auch Mixed Reality (MR) umfasst.

Das Microsoft OpenXR Plugin erleichtert Entwicklern die Arbeit innerhalb von Unity, indem es eine reibungslose Integration
des OpenXR-Standards ermöglicht. Hierdurch erhalten Entwickler Zugang zu einer breiten Palette von Funktionen und
Möglichkeiten, die durch den OpenXR-Standard standardisiert sind, inklusive plattformübergreifender Kompatibilität sowie
optimierter Leistung für XR-Anwendungen.

Die Relevanz des Microsoft OpenXR Plugins für die AR-Entwicklung liegt in seiner Fähigkeit, eine konsistente Entwicklungsumgebung
für XR-Anwendungen innerhalb von Unity zu schaffen. Durch die Nutzung dieses Plugins können AR-Entwickler die Vorteile
des OpenXR-Standards voll ausschöpfen, einschließlich verbesserte Kompatibilität, Leistung und Zukunftssicherheit ihrer Anwendungen.

Zusätzlich erlaubt das Plugin den Entwicklern den Zugriff auf spezifische Funktionen und Optimierungen, die von Microsoft
speziell für die AR-Entwicklung entwickelt wurden. Dies beinhaltet beispielsweise Features zur Verbesserung der
AR-Umgebungserkennung, Handhabung von Eingaben sowie Optimierung der Grafikleistung für AR-Anwendungen.\footnote{Khronos Group \cite{OpenXR Plugin}}

\subsubsection{Integrationsprozess der Plugins}
Zur Integration der genannten Plugins in
den Unity Editor, um diese verwenden zu können, wird das externe Tool namens \textit{Mixed Reality Feature Tool} von Microsoft verwendet. Dieses Tool
fungiert als umfassende Sammlung von Plugins und Erweiterungen im Bereich der Augmented und Virtual Reality Entwicklung
für Unity. Neben den erwähnten Plugins umfasst es weitere wichtige Erweiterungen, darunter:
\begin{itemize}
    \item Azure Mixed Reality Services
    \item Experimental
    \item Mixed Reality Toolkit
    \item Plattform Support
    \item Spatial Audio
    \item World Locking Tools
    \item Weitere Funktionen (Other features)
\end{itemize}

Um die Integration durchzuführen, müssen innerhalb des \textit{Plattform Support} Bereichs des \textit{Mixed Reality Feature Tools}
sowohl das \textit{Mixed Reality OpenXR Plugin} als auch das gesamte MRTK3-Paket ausgewählt werden. Nach der Auswahl
dieser Pakete ist es erforderlich, sie zu validieren und anschließend in das Unity-Projekt zu importieren.

Dieser Prozess gewährleistet eine erfolgreiche Integration der benötigten Plugins und Erweiterungen, die für die Entwicklung
von Augmented Reality-Anwendungen in Unity von entscheidender Bedeutung sind.

\subsection{Wahl des Modellierungsprogramm} \label{sec:wahlblender} \marginpar{\small\(\rightarrow\) {\tiny LAMPEL}}
Durch Unity und die Open-Source-lastige Modellierungscommunity gibt es verschiedene Möglichkeiten, um an die 3D-Modelle
zu gelangen, die für die Anwendung der Szenarien benötigt werden.

Eine Möglichkeit besteht darin, auf weit verbreitete 3D-Modelle auf sogenannten \textit{Asset Stores} zurückzugreifen,
wie zum Beispiel dem von Unity selbst. Dort können sie entweder gekauft oder kostenlos heruntergeladen werden. Die
verschiedenen Angebote unterscheiden sich in der Texturqualität und dem Modellierungsaufwand.
\footnote{Unity \cite{Asset Store}}

Eine weitere Möglichkeit besteht darin, die Modelle selbst in einem Modellierungsprogramm zu entwerfen. Diese Variante
ist zwar zeitintensiver und aufwendiger, führt jedoch zu maßgeschneiderten Modellen, die den Wünschen und Anforderungen
des Projekts entsprechen. Es gibt verschiedene Programme, die die Anforderungen erfüllen würden. Beispielsweise würden
Blender \footnote{Blender \cite{Blender Allgemein}}, 3ds Max \footnote{Autodesk \cite{3DS Max}} von Autodesk oder
Cinema 4D \footnote{Maxon \cite{Cinema 4D}} von Maxon in Frage kommen. Diese Programme unterscheiden sich nur hinsichtlich
der Bedienung, aber der eigentliche Zweck ist gleich.

Nach Absprachen innerhalb des Teams wurde beschlossen, das Modellierungsprogramm Blender zu benutzen, da eines der
Teammitglieder bereits gewisses Wissen in der Bedienung und Modellierung mit diesem Programm angehäuft hat. Nicht nur
das war der ausschlaggebende Punkt, sondern auch die große Benutzergemeinschaft in Blender mit zahlreichen Tutorials
und Hilfestellungen für auftretende Herausforderungen.

Obwohl diese Variante einen höheren Zeitaufwand und zusätzliche Komplexität des Projekts mit sich bringt, erhält man
am Ende eine Sammlung von Objekten, die vollständig den Anforderungen und Wünschen entspricht. Zudem ist dies die
kostengünstigste Methode, da der Kauf von Modellen oder einer kostenpflichtigen Modellierungsplattform vermieden werden
konnte, um hohe Kosten zu vermeiden.

\subsubsection{Wie funktioniert Blender im Allgemeinen?}
Die Applikation Blender ist sehr komplex. Daher werden in der folgenden Beschreibung nur die Schlüsselaspekte und die
Funktionalität von Blender für unseren speziellen Anwendungsbereich hervorgehoben.

\begin{itemize}
    \item \textbf{Benutzeroberfläche und Interaktion}\\
    Die Benutzeroberfläche von Blender ist hoch anpassbar, obwohl sie komplex gestaltet ist. Sie enthält 3D-Modelle,
    Ansichten, Fenster und Panels. Benutzer interagieren mit Objekten und Werkzeugen über Maus- und Tastaturbefehle.
    Die Effizienz der Arbeit und die Modellierungsdauer hängen stark von der Erfahrung des Benutzers ab, insbesondere
    von der Nutzung von Shortcuts und Hotkeys.\footnote{Blender \cite{Benutzeroberfläche}}

    \item \textbf{3D-Modellierung}\\
    Blender ermöglicht die Erstellung von 3D-Modellen durch die Verwendung von Primitiven wie Würfeln, Kugeln,
    Flächen und Kurven. Diese können bearbeitet und modifiziert werden, um komplexe Formen zu erstellen.
    Modellierungswerkzeuge wie Extrusion, Verschiebung, Skalierung und Rotation stehen zur Verfügung.
    \footnote{Blender \cite{Toolbar}}

    \item \textbf{Materialien und Texturen}\\
    Um realistische Oberflächen zu erzeugen, können Materialien erstellt und Texturen auf Objekte angewendet werden.
    Blender ermöglicht die Feinanpassung von Materialeigenschaften wie Diffusreflexion, Glanz, Transparenz und Emission.
    \footnote{Blender \cite{Materials}}

    \item \textbf{Gemeinschaft und Ressourcen}\\
    Blender hat eine engagierte Benutzergemeinschaft, die umfassende Dokumentation, Tutorials und Foren
    bereitstellt. Diese Ressourcen erleichtern die Einarbeitung und Problemlösung.
\end{itemize}

Blender wird im gesamten Projekt eingesetzt, angefangen beim Entwurf und der Erstellung des Hauptmenüs bis hin zur
Gestaltung jedes einzelnen Objekts. Es wird hauptsächlich für die digitale Modellierung der wichtigsten täglichen
Gegenstände von Schülern verwendet. Das Ziel besteht darin, am Ende eine umfangreiche Sammlung von Objekten zu haben,
um den Benutzern eine vielfältige und zahlreiche Auswahl an verschiedenen Modellen zu bieten.
