\chapter{Zusammenfassung und Abschluss}

\section{Ergebnis}
Die Umsetzung des Projekts verlief erfolgreich und alle geplanten Ziele wurden erreicht. Als Ergebnis stehen eine
Augmented Reality-Anwendung und eine Website zur Verfügung, die im Unterricht als Lehrmittel eingesetzt werden können
und auch am Tag der Offenen Tür genutzt werden können, um diesen interessanter zu gestalten.

\section{Abnahme}
Die Abnahme des Projekts erfolgte am 03.04.2024 mit der Abgabe dieser Diplomarbeit. Die HTBLuVA Wiener Neustadt plant,
diese Applikation in naher Zukunft im Unterricht sowie am Tag der offenen Tür einzusetzen.

\section{Zukunft}
Die Zukunftsperspektive bietet verschiedene Möglichkeiten zur Weiterentwicklung und Erweiterung der vorliegenden
Projektarbeit. Ein möglicher Ansatz besteht darin, das vorhandene Wissen und die Technologie an ein nachfolgendes Team
weiterzugeben, um zusätzliche Anwendungsszenarien zu entwickeln. Durch die Initialisierung des Projekts auf der HoloLens
2 durch das aktuelle Projektteam steht den Nachfolgern eine solide Grundlage zur Verfügung. Diese bietet bereits eine
Vielzahl von behobenen Problemen und dokumentierten Lösungsansätzen. Dadurch können Zeit und Ressourcen gespart werden,
und der Fokus kann verstärkt auf die Erweiterung und Verbesserung des bestehenden Systems gelegt werden.

Während der Projektarbeit wurden im Team verschiedene Ideen für potenzielle Anwendungsszenarien diskutiert und konzipiert.
Eine vielversprechende Idee wurde identifiziert, die das Spektrum der Anwendungsmöglichkeiten der HoloLens 2 erweitern könnte:

\begin{itemize}
    \item \textbf{PC-Zusammenbau-Simulator:} Dieses interaktive Szenario ermöglicht es dem Benutzer, den detaillierten
    Zusammenbau eines PCs virtuell durchzuführen. Der Benutzer kann aus einem umfangreichen Katalog verschiedener Bauteile
    wählen und diese gemäß den Anweisungen in den PC einbauen. Das Szenario führt den Benutzer durch die einzelnen Schritte 
    des Zusammenbaus und erklärt die Funktionen der verschiedenen Komponenten.
\end{itemize}