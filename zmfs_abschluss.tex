\chapter{Zusammenfassung und Abschluss}

\section{Ergebnis}
Die Implementierung des Projekts wurde erfolgreich abgeschlossen. Alle definierten Ziele wurden gemäß den vorgegebenen
Spezifikationen erreicht. Das resultierende Produkt ist eine Augmented Reality (AR)-Applikation sowie eine begleitende
Webseite. Die AR-Anwendung bietet eine innovative Möglichkeit, Lehrinhalte zu vermitteln, indem sie eine interaktive und
immersivere Lernerfahrung ermöglicht. Durch die Integration von virtuellen Elementen in die reale Umgebung können komplexe
Konzepte auf anschauliche Weise vermittelt werden. Die Webseite dient als ergänzendes interaktives Element, um dem Benutzer
eine bessere Interaktion mit der realen Welt zu ermöglichen und somit das Verständnis des veranschaulichten Anwendungsszenarios
zu vertiefen.

Das entstandene Produkt bietet vielfältige Einsatzmöglichkeiten im Bildungskontext. Es kann nicht nur als Lehrmittel im
regulären Unterricht verwendet werden, sondern auch als Attraktion während Veranstaltungen wie dem Tag der Offenen Tür.
Durch die Integration moderner Technologien wird die Lernumgebung bereichert und die Aufmerksamkeit der Zielgruppe auf
innovative Weise auf sich gezogen. Dies trägt dazu bei, das Interesse an den behandelten Themen zu wecken und das Lernengagement
zu steigern.

Die erfolgreiche Umsetzung des Projekts markiert einen bedeutenden Fortschritt in der Verknüpfung von technologischen
Innovationen mit Bildungsprozessen. Die erzielten Ergebnisse bieten nicht nur einen Mehrwert für die Lehrenden und Lernenden
an der HTBLuVA Wiener Neustadt, sondern stellen auch einen Beitrag zur Weiterentwicklung pädagogischer Ansätze dar, die
auf den Einsatz digitaler Medien und interaktiver Technologien abzielen.

\section{Abnahme}
Die formelle Abnahme des Projekts fand am 20.03.2024 mit der Einreichung dieser Diplomarbeit statt. Die HTBLuVA Wiener
Neustadt plant, die entwickelte Applikation in absehbarer Zeit in ihren Lehrplan zu integrieren und bei öffentlichen
Veranstaltungen wie dem Tag der offenen Tür einzusetzen.

Die Entscheidung der HTBLuVA Wiener Neustadt, die Anwendung in ihr Bildungsangebot aufzunehmen, unterstreicht die Relevanz
und den Mehrwert des Projekts für die Bildungseinrichtung. Die geplante Integration in den Lehrplan zeigt das Vertrauen
in die Qualität und Effektivität der entwickelten Lösung.

Die Abnahme des Projekts durch die Bildungsinstitution markiert einen Meilenstein in der Validierung der Arbeit und der
erzielten Ergebnisse. Es ist wichtig, dass die entwickelten Produkte den Anforderungen und Erwartungen der Zielgruppe
entsprechen und den angestrebten Bildungszielen dienen. Dabei sollte eine klare und logische Strukturierung der Informationen
gewährleistet sein, um eine einfache Verständlichkeit zu gewährleisten.

\section{Zukunft}
Die Zukunftsperspektive bietet verschiedene Möglichkeiten zur Weiterentwicklung und Erweiterung der vorliegenden Projektarbeit.
Ein möglicher Ansatz besteht darin, das vorhandene Wissen und die Technologie an ein nachfolgendes Team weiterzugeben,
um zusätzliche Anwendungsszenarien zu entwickeln. Durch die Initialisierung des Projekts auf der HoloLens 2 durch das aktuelle
Projektteam steht den Nachfolgern eine solide Grundlage zur Verfügung. Diese bietet bereits eine Vielzahl von behobenen
Problemen und dokumentierten Lösungsansätzen. Dadurch können Zeit und Ressourcen gespart werden, und der Fokus kann
verstärkt auf die Erweiterung und Verbesserung des bestehenden Systems gelegt werden.

Während der Projektarbeit wurden im Team verschiedene Ideen für potenzielle Anwendungsszenarien diskutiert und konzipiert.
Eine vielversprechende Idee wurde identifiziert, die das Spektrum der Anwendungsmöglichkeiten der HoloLens 2 erweitern könnte:

\begin{itemize}
    \item \textbf{PC-Zusammenbau-Simulator}: Dieses interaktive Szenario ermöglicht es dem Benutzer, den detaillierten
    Zusammenbau eines PCs virtuell durchzuführen. Der Benutzer kann aus einem umfangreichen Katalog verschiedener Bauteile
    wählen und diese gemäß den Anweisungen in den PC einbauen. Das Szenario führt den Benutzer durch die einzelnen Schritte
    des Zusammenbaus und erklärt die Funktionen der verschiedenen Komponenten.
\end{itemize}